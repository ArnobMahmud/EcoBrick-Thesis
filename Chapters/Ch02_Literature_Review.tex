\chapter{Literature Review }
A composite is a material manufactured from two or more discrete materials that are insoluble at a macroscopic level but together create enhanced properties. Textile means fibrous goods (like cotton, polyester, denim, woven, or knitted fabric) used as reinforcement in the composite. The matrix is the continuous phase that binds the shredded textiles together and surrounds them, creating a solid form after curing. Sustainable construction material development has been gaining worldwide traction in the aftermath of increasing environmental awareness and infrastructure demands in urban areas. In this scenario, several studies have explored new applications of industrial and agricultural wastes for improving material efficiency, eliminating pollution, and lowering construction costs. This chapter discusses an elaborate review of the existing literature on (i) textile waste generation and disposal, (ii) composite materials through recycled textiles, (iii) resin-based composite applications, and (iv) alternative bricks with waste materials. The purpose is to create a scholarly background and establish research gaps to fill in the context of Bangladesh as well as globally. 

\section{Textile Waste: Global Trends and Bangladesh Scenario}

Worldwide, the textile and fashion industry are one of the most resource-consuming sectors that generate more than 92 million tons of waste each year. In Bangladesh, the RMG sector, although earning about 84\% of the country's export revenue, produces about 400,000 tons of solid textile wastes yearly. These cotton, polyester, denim, and mixed woven and knitted fabric-based wastes are generally disposed of in landfills or incinerated without any formal recycling system, which contributes to serious environmental issues like groundwater pollution and greenhouse gas emissions. \\

\noindent Research (by Hasan et al., 2020 and Rahman and Ahsan, 2019) shows that most factories in Dhaka, Narayanganj, and Chattogram lack systematic textile waste management protocols, though up to 25\% of production material is wasted per garment unit.
\section{Composite Materials from Recycled Textiles }

Textile waste, particularly from the garment industry, poses a serious environmental challenge due to its slow degradation and volume. Researchers have shown that chopped textile can be incorporated into construction materials to enhance properties such as flexibility, insulation, and toughness (Yin et al., 2021). Waste textile are increasingly considered viable reinforcements in polymer or gypsum matrices for brick and panel production, owing to their fibrous nature and energy-absorbing capacity (Miah et al., 2020). \\

\noindent A composite textile in resin matrix is a fiber-reinforced polymer (FRP) material, where shredded textile (reinforcement) are embedded in a resin matrix (binder), resulting in a durable and lightweight construction material. Several international studies have examined the mechanical and thermal applications of recycled textiles in construction. Shredded textiles such as cotton and polyester have been successfully used in insulation boards, low-strength concrete, geo-textiles, and polymer-based composites. Shredded cotton improves the ductility of cement composites, while synthetic shredded textiles enhance dimensional stability and moisture resistance (Park et al., 2018). Studies have highlighted that composite bricks made from alternative binders can perform competitively in terms of mechanical and thermal properties when designed appropriately (Ghosh et al., 2020; Zhang et al., 2019). 
\section{Resin-Bonded Composites in Construction}
The employment of resin-based systems as binders has gained prominence in the formulation of new construction materials. Some of the most common ones used are polyester resin, epoxy resin, and plaster of Paris (POP) with each of them possessing different mechanical and chemical characteristics for composite development. \\

\noindent Polyester resin, due to its low price, simple processing, and good mechanical properties, is a popular choice. It has seen widespread application in polymer concrete, fiber reinforced panels, and composite tiles. Polyester resin, when blended with shredded textile waste, especially synthetic shredded textiles like polyester and natural shredded textiles like cotton, offers good adhesion, forming a stiff matrix that increases compressive strength and dimensional stability. Epoxy resin, being costlier than polyester resin, provides higher bonding strength, minimal shrinkage, and very good chemical resistance. This makes it highly suitable for structural applications where the requirement is high durability and resistance to moisture. Research has established that epoxy-bound textile composites have better mechanical integrity and service life when subjected to environmental stress than with other binding systems. Studies suggest that epoxy-based bricks show higher compressive and flexural strength compared to polyester-based ones due to better matrix-fiber bonding (Hussain et al., 2021). Plaster of Paris (POP), which mainly consists of calcium sulfate hemihydrate, is extensively utilized in low-load applications owing to its quick setting and good finish. As a partial binder or filler in composite bricks, POP has the potential to enhance thermal insulation and surface finish. Nevertheless, it needs careful formulation to prevent brittleness and water solubility. Research indicates that combining POP with additives or shredded textiles can enhance its performance and durability in composite bricks (Ahmed et al., 2018). \\

\noindent In this study, all the three matrices - polyester resin, epoxy resin, and POP were used in different proportions to balance the strength, curing time, and economy of the composite bricks. Their compatibility with different kinds of textile waste (cotton, polyester, denim, woven, and knitted fabrics) was investigated systematically to arrive at the most suitable formulation for structural and environmental sustainability. \\

\noindent Several studies have focused on evaluating compressive strength, flexural strength, surface hardness, thermal conductivity, and water absorption of composite bricks. The ASTM standards such as:

\begin{enumerate}
	\item ASTM C67 for compressive strength,
	\item ASTM C293 for flexural strength,
	\item ASTM C642 for density,
	\item ASTM D2240 for surface hardness, and
	\item ASTM D570 for water absorption are commonly adopted for these evaluations.
\end{enumerate}
