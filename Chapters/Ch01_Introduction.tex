\chapter{Introduction}

The growing global population, along with unparalleled technological progress and economic growth, has created a steep rise in energy needs, consumption of resources, and production of waste and greenhouse gases. Such pressures are profoundly affecting ecological balance and planetary sustainability. As the Global Footprint Network (2018) has pointed out, if human beings persist in using resources at the current level, we would need the equivalent of three Earths to satisfy our yearly resource needs -- a grim reflection of unsustainable consumption patterns [1].\\

\noindent One of the significant drivers of this over-consumption is the international textile market, powered by fast fashion, increasing living standards, and mass production. As fashion cycles become shorter and apparel more disposable, the amount of textile waste produced annually continues to increase exponentially. With recent worldwide estimates suggesting that the textile industry alone accounts for around 55\% of overall solid waste globally, much of which finds its way into landfills or incinerators and generates air, soil, and water pollution [2]. In the meantime, brick production, a fundamental sector of the building sector, continues to be among the biggest drivers of air pollution, especially in developing nations where obsolete technologies use trash, tires, plastics, and textiles as fuel and spew toxic emissions into the environment [3].\\

\noindent Aside from air pollution, the environmental impact of the textile industry is colossal. The World Bank (2020) places the textile and clothing industry as the world's secondlargest industrial polluter after the oil and petrochemical industries. It is also about 14\% of total landfill waste, and contributes to 20\% of worldwide industrial water pollution due to dyeing and treatment activities [4]. The textile production also has a sizeable greenhouse gas emission, which is larger than those generated by international flights and shipping combined [5]. \\

\noindent This is exacerbated by the linear economic model that controls textile production and disposal -- a framework in which resources are extracted, used, and dumped with little recycling. Yet research shows that as much as 90\% of post-consumer textile waste is reusable or recyclable [6]. But because of limited awareness, infrastructure, and scalable recycling methods, the majority of textile waste is lost to landfills, where it emits methane -- a powerful greenhouse gas -- as it breaks down.\\

\noindent To combat this, the circular economy concept has come to the fore as an attractive model, with a focus on waste reduction through reuse, recycling, re-manufacturing, and up-cycling. Textile waste, previously an environmental burden, is being re-think as a valuable input for industrial purposes like furnishings, insulation material, and now, more and more, construction materials. One of the most hopeful strategies is incorporating shredded textile waste into composite building materials such as bricks, providing a twofold solution to environmental degradation and resource depletion.\\

\noindent Comparative research indicates that recycling textiles for industrial purposes is 20--100 times more environmentally friendly than incineration or chemical recycling, mainly because of decreased emissions and energy demand [6]. The advantages are the decreased reliance on virgin raw materials, reduced building costs, and the considerable decrease in landfill volume and CO$_2$ emissions. The crisis needs to be addressed with a systemic re-imagination of both construction practices and textile waste management. Conventional practices of land filling and incineration are no longer effective or sustainable. Rather, new, interdisciplinary solutions need to be embraced -- encompassing material science, environmental engineering, and circular economy principles. This study thus explores how chopped textile waste -- such as cotton, polyester, denim, woven and knitted fabrics -- can be successfully re-utilized in the form of composite bricks with the use of binders like epoxy resin, polyester resin (PET), and plaster of Paris (POP). The research not only alleviates the environmental load of textile and construction waste but also offers a scalable, replicable, and sustainable solution with worldwide applicability. \\

\noindent The world construction industry is growing at a high rate, fueled mainly by population increase, rapid urbanization, and infrastructural expansion in both developed and developing countries. The worldwide construction industry was worth USD 6.4 trillion as of 2020 and is expected to grow to USD 14.4 trillion by 2030, almost doubling in a single decade due to rising demand for residential and infrastructural development [6]. This boom, nevertheless, is also heightening the use of natural resources. Consequently, researchers are in a race to explore sustainable options in construction and furniture materials, especially from waste streams and recycled sources, in an effort to minimize environmental footprint and resource utilization [6]. \\

\noindent The disposal of unmanaged industrial waste, particularly from the textile and construction industries, poses an acute environmental issue. The reuse of such waste in building materials is not only economically viable but also promotes environmental sustainability. Interestingly, the textile and construction industries are jointly responsible for approximately 12\% of worldwide CO$_{2}$ emissions, calling for revolutionary measures in waste minimization and material development [7]. \\

\noindent There has been an increasing amount of research investigating the incorporation of different textile wastes in building composites. Researchers have worked with, for example, textile cutting waste [8], sludge from textile effluent treatment plants [9], cotton micro-dust waste [10], polyester/cotton blend fabric waste [11], glass wool insulation waste [12], and cotton stalk fiber waste [13]. Outcomes from these studies all report improvements in thermal insulation (by as much as 3--4\%), acoustic dampening, and reduction in material cost -- rendering these bricks suitable for sustainable construction [14]. \\

\noindent Concurrently, there were a number of high-profile innovations in fabric waste upcycling. Kamble and Behera created eco-friendly furniture panels from cotton shoddy, waste glass fiber preforms, and jute-based nonwoven sheets with a 5\% enhancement in mechanical properties through a 3\% cellulosic filler addition by volume [15]. Marlet and her Fab-BRICK studio used textile scraps and eco-friendly starchbased glues and mechanical compression to create modular bricks, with 4\% and 7\% enhancements in tensile and flexural strength, respectively [16] [17] [18] [19]. Andreu also created decorative bricks through acrylic selvedge waste with water-based acrylic resin, which resulted in a 2--3\% improvement in thermal insulation [20]. \\

\noindent Other innovations are Ackerman’s carbon-neutral textile bricks, which are built from fabric remnants and clothing accessories such as buttons and zippers, achieving significant CO${_2}$ emission savings and better thermal regulation [21]. These innovations indicate that the combination of textile-based bricks and insulation materials can drastically reduce heating and cooling requirements -- facilitating net-zero energy building. Also, E. Kagitci achieved the use of 100\% cotton, silk, and viscose textilewaste bonded by starch-based adhesive in textile bricks, estimating up to 30\% cost savings for environmentally friendly construction [22]. D. Trajkovic et al. investigated the application of polyester garment cuttings waste in insulation bricks and reported 10\% better fire resistance, 22\% higher moisture resistance, 25\% better sound insulation, greater durability, and longer product lifespan, making them ideal for partition walls, furniture, and ornamental architectural features [23]. \\

\noindent In spite of these encouraging advances, there exist notable gaps in research -- especially regarding the relative performance of different binders (e.g., epoxy resin, polyester resin, and plaster of Paris) in textile-reinforced bricks. Comparatively few studies have directly evaluated the mechanical, thermal, or durability-related results when different resin matrices are applied with different types of fabrics such as cotton, polyester, denim, woven and knitted fabrics. Binder-to-textile ratios, environmental exposure, cost-effectiveness, scalability, and life cycle sustainability are among the underexplored factors [24] [25]. Thus, the current research seeks to fill this gap through the development and assessment of composite bricks produced from shredded textile waste adhered with a range of resins and additives. This research concentrates on mechanical strength (compression, tensile, and flexural), thermal insulation, and resistance to moisture, comparing these with conventional clay bricks to determine their feasibility  [26]. The goal is to create a scalable, low-cost, and sustainable construction solution that resonates with the principles of the circular economy while helping reduce the carbon footprint of the textile and construction sectors.

\section{Purpose and Significance of the Study}
The significance of this study lies in its multifaceted contribution: reducing the ecological footprint of textile waste, minimizing the environmental burden of traditional brick kilns, and introducing a cost-effective alternative building material tailored for the socio-economic and climatic context of Bangladesh. It also serves as a strategic alignment with national and global sustainable development goals, particularly those relating to waste management, climate action, affordable housing, and industry innovation. This study introduces a novel solution that addresses these twin challenges: the development of composite bricks reinforced with shredded textile waste (including cotton, polyester, denim, woven, and knitted fabrics) bound with resin. This alternative material not only diverts textile waste from landfills but also reduces dependency on clay extraction and fossil fuel consumption associated with traditional brick kilns. The use of thermosetting or bio-based resin as a binder enables effective encapsulation and solidification of shredded textiles, yielding bricks with favorable mechanical and thermal properties. The significance of this research lies in its multidimensional value: it proposes a technologically feasible, economically viable, and environmentally sustainable construction material. Furthermore, it aligns with Bangladesh's national development goals and global commitments, including the UN Sustainable Development Goals (SDGs) particularly those relating to responsible consumption, climate action, and sustainable cities. 
\section{Aim of the Research}
The principal aim of this research is to assess the technical, environmental, and economic feasibility of using recycled textile waste composites in the production of bricks with special emphasis on enhancing structural behavior, water absorbency performance, and sustainability. The research endeavors to set a new trend in material recycling, enabling the transition to green construction practices in Bangladesh. 
\section{Research Questions}
This study is guided by the following central research questions:

\begin{enumerate}

\item Primary Research Question: 
Can shredded textile waste, when combined with resin, be effectively transformed into structurally viable and sustainable composite bricks suitable for the construction industry globally? 

\item Secondary Research Questions: 
\begin{itemize}
    \item What are the optimal material compositions (fiber types, ratios, and resin types) to achieve favorable physical and mechanical properties in the bricks?

    \item How do resin-bound textile composite bricks compare to traditional clay bricks in terms of compressive strength, durability, bending and water absorption?

    \item What environmental benefits-such as carbon footprint reduction and waste diversion-are associated with using these composite bricks?

    \item What are the potential challenges in manufacturing, standardization, and market acceptance of these bricks within the Bangladeshi context?
\end{itemize}

\end{enumerate}
\section{Research Limitations}
Although the research presents an innovative and environmentally compelling approach, several limitations may affect its scope and generalizability:

\begin{enumerate}
	\item Material Heterogeneity: Textile waste, particularly from mixed fiber sources such as denim, knitted, and woven fabrics, varies in texture, tensile strength, and dye content, which may influence consistency in composite formation.

	\item Resin Selection: The study is limited to specific types of resin (e.g., polyester, epoxy, or bio-based resins) due to availability and cost constraints. Resin toxicity and curing requirements may also pose environmental and safety considerations.

	\item Scale of Production: Fabrication is limited to a laboratory or pilot scale. Industrial-scale production, cost modeling, and long-term field testing remain outside the scope of this study.

	\item Time Constraints: Due to the academic calendar, the study does not include long-term environmental exposure or weathering tests, which are critical to validate outdoor performance.

	\item Regulatory Hurdles: Introducing a non-traditional material into mainstream construction will require alignment with local building codes and may encounter resistance from stakeholders unfamiliar with composite technologies.
\end{enumerate}