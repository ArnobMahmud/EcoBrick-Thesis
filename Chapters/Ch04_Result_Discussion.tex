\chapter{Result and Discussion}
This chapter discusses and presents the comprehensive performance testing of the newly developed composite bricks fabricated using shredded textile waste integrated with various binder systems. In an effort to promote sustainable and eco-friendly construction materials, three distinct formulations were designed. Each formulation contained an identical quantity of chopped textile waste (100 grams), which was uniformly mixed with different types of binders to evaluate their influence on brick properties.  The first formulation utilized epoxy resin and hardener, known for their excellent bonding and mechanical strength. \\

\noindent The second composition included polyethylene terephthalate (PET) resin along with appropriate catalysts, selected for its thermoplastic characteristics and chemical resistance. The third mixture employed a plaster of Paris (POP) and cement blend, offering a more conventional and cost-effective alternative. \\

\noindent All composite mixtures were thoroughly blended and cast in a standardized rectangular brick mold with dimensions of 200 mm $\times$ 100 mm $\times$ 60 mm. After proper curing, the fabricated bricks underwent a series of standardized tests to evaluate their mechanical and physical properties.  \\

\noindent These tests included compressive strength, surface hardness, flexural strength, water absorption, and density, conducted in accordance with relevant ASTM standards. The primary objective of this testing phase was to assess the structural viability, durability, and potential application of these textile-reinforced composite bricks in real-world construction scenarios. \\

\noindent Through this comparative evaluation, insights were gained regarding the most suitable binder for enhancing the mechanical performance and sustainability of textile waste based bricks.

\section{Test Result and Discussion}

% --------------- 4.1.1 --------------------%
\subsection{Compressive Strength Test Result}
\begin{table}[h!]
	\renewcommand{\arraystretch}{2} % row height
	\setlength{\tabcolsep}{10pt} % column padding
	\begin{tabular}{
		|>{\centering\arraybackslash}m{3cm}|
		>{\centering\arraybackslash}m{3cm}|
		>{\centering\arraybackslash}m{3cm}|
		>{\centering\arraybackslash}m{3cm}|
		}
		\hline
		\rowcolor{gray!20}
		Sample Code. & Load at Failure (kN) & Area under Load (mm$^2$) & Compressive Strength (N/(mm$^2$) or MPa) \\ \hline
		CRB-01       & 110                  & \multirow{3}{*}{20000}   & 5.50                                     \\
		\cline{1-2} \cline{4-4}
		CRB-02       & 130                  &                          & 6.50                                     \\
		\cline{1-2} \cline{4-4}
		CRB-03       & 70                   &                          & 3.50                                     \\
		\cline{1-2} \cline{4-4}
		\hline
	\end{tabular}
	\caption{Compressive Strength Test Result}
	\label{tab:placeholder}
\end{table}

\begin{figure}[H]
	\begin{minipage}{1\textwidth}
		\centering
		\fbox{\includegraphics[height=10cm, width=1\linewidth]{Assets/27.png}}
		\caption{Compressive Strength Test Result}
	\end{minipage}
\end{figure}

\begin{enumerate}
	\item CRB-01 (Epoxy Brick) attained a compressive strength of 5.50 MPa, which signifies good mechanical behavior and structural soundness. Epoxy resin is a rigid and highly cross linked matrix that strongly bonds with shredded textiles. The bonding action improves the inner structure of the brick, enabling it to sustain compressive loads nicely. The inherent brittleness of epoxy, though, can cause sudden failure at high stress in the presence of voids or micro cracks. Nevertheless, CRB-01 is still a good choice for light structural applications where strength and resistance to moisture are required.

	\item CRB-02 (Polyester Brick) had the greatest compressive strength among the three, at 6.50 MPa. This may be due to the fact that unsaturated polyester resin is a more flexible material that distributes stress more evenly and retains better adhesion with different textile reinforcements. The resin-shredded textile mix formed a dense, well consolidated matrix that could resist higher compressive forces. Consequently, CRB-02 is the most structurally sound of the test bricks and is highly suitable for application in lightweight structural members or partition walls.

	\item CRB-03 (Plaster of Paris Brick) had the lowest compressive strength at 3.50 MPa. Plaster of Paris, while being malleable and light, is inherently porous and brittle. Despite the incorporation of cement and shredded textiles, its compressive stress resistance capacity is still low. The shredded textiles can enhance crack resistance to some extent, but they cannot make up for the POP matrix's weakness. CRB-03 is thus most ideal for non-load-bearing purposes like decorative panels or interior partitions where mechanical strength is not important.
\end{enumerate}

% --------------- 4.1.2 --------------------%
\subsection{Density Test Result}
\begin{table}[h!]
	\renewcommand{\arraystretch}{2} % row height
	\setlength{\tabcolsep}{5.5pt} % column padding
	\begin{tabular}{
		|>{\centering\arraybackslash}m{1.715cm}|
		>{\centering\arraybackslash}m{1.715cm}|
		>{\centering\arraybackslash}m{1.715cm}|
		>{\centering\arraybackslash}m{1.715cm}|
		>{\centering\arraybackslash}m{1.715cm}|
		>{\centering\arraybackslash}m{1.715cm}|
		>{\centering\arraybackslash}m{1.715cm}|
		}
		\hline
		\rowcolor{gray!20}
		Sample Code & Length (m)           & Width (m)            & Height (m)             & Volume (m$^3$)            & Weight (Kg) & Density (Kg/m$^3$) \\ \hline
		CRB-01      & \multirow{3}{*}{0.2} & \multirow{3}{*}{0.1} & \multirow{3}{*}{0.027} & \multirow{3}{*}{0.00054 } & 0.50        & 925.92             \\
		\cline{1-1} \cline{6-7}
		CRB-02      &                      &                      &                        &                           & 0.56        & 1037.04            \\
		\cline{1-1} \cline{6-7}
		CRB-02      &                      &                      &                        &                           & 0.49        & 907.41             \\
		\cline{1-1} \cline{6-7}
		\hline
	\end{tabular}
	\caption{Density Test Result}
	\label{tab:placeholder}
\end{table}

\begin{figure}[H]
	\begin{minipage}{1\textwidth}
		\centering
		\fbox{\includegraphics[height=10cm, width=1\linewidth]{Assets/28.jpg}}
		\caption{Density Test Result}
	\end{minipage}
\end{figure}

\begin{enumerate}
	\item CRB-01 exhibited a density of 925.92 kg/m³, placing it in the mid-range among the three samples. The use of epoxy resin as the binder contributes to a relatively compact structure due to its high adhesive properties, enabling better encapsulation of textile waste particles. However, the density remains lower than that of CRB-02, indicating that while epoxy provides decent structural cohesion, it may leave some voids or have lower material packing efficiency compared to thermoplastics. The moderate density of CRB-01 suggests a balanced trade-off between weight and structural performance, making it suitable for non-load-bearing applications where lightweight and durability are desired.
	\item CRB-02 achieved the highest density value of 1037.04 kg/m³, clearly indicating the superior compactness of the material matrix. PET resin, being a thermoplastic, likely filled the voids more effectively and formed a denser matrix when combined with textile waste. The higher density implies a reduced porosity and greater material integrity, which may translate into better mechanical performance in terms of compressive and flexural strength. This makes CRB-02 potentially more suitable for structural applications where higher material density is advantageous, although thermal insulation properties may be compromised due to the reduced air pockets.
	\item CRB-03 recorded the lowest density of 907.41 kg/m³, suggesting a less compact and more porous internal structure compared to the other two variants. This outcome is likely due to the inherent lightweight nature of the plaster of Paris and the air entrainment typically associated with POP-cement mixtures. Although this lower density may result in reduced compressive strength, it could offer better thermal insulation and lower transportation costs. CRB-03 may therefore be more suited to interior partitioning, temporary structures, or applications where load-bearing capacity is not the primary concern but lightweight and insulation are desirable.
\end{enumerate}

% --------------- 4.1.3 --------------------%
\subsection{Surface Hardness Test Result}
\begin{table}[h!]
	\renewcommand{\arraystretch}{2} % row height
	\setlength{\tabcolsep}{20pt} % column padding
	\begin{tabular}{
		|>{\centering\arraybackslash}m{6cm}|
		>{\centering\arraybackslash}m{6cm}|
		}
		\hline
		\rowcolor{gray!20}
		Sample Code. & Surface Hardness (Shore D) \\ \hline
		CRB-01       & 78                         \\  \hline
		CRB-02       & 72                         \\  \hline
		CRB-02       & 58                         \\  \hline
	\end{tabular}
	\caption{Surface Hardness Test Result}
	\label{tab:placeholder}
\end{table}
\begin{enumerate}
	\item CRB-01 (Epoxy Brick) had the best surface hardness with a reading of 78 on the Shore D scale. This suggests a hard, stiff outer layer that is resistant to indentation and abrasion. The high hardness of epoxy is due to its robust cross-linked molecular structure, which makes the brick viable for use where surface durability matters. The high Shore D rating indicates the material's resistance to wear and mechanical damage, implying that CRB-01 may be suitable for use on exposed or high-contact surfaces like flooring tiles, wall panels, or cladding.
	\item CRB-02 (Polyester Brick) had a somewhat lower surface hardness of 72 Shore D, which is still in the high-performance range. Unsaturated polyester resin creates a tough matrix, although one that is a little less stiff than epoxy. The effect is a surface that retains good hardness and scratch resistance but permits a little more give under impact. This makes CRB-02 an excellent option for moderate-use construction surfaces, particularly where a compromise between toughness and workability is needed.
	\item 3. CRB-03 (Plaster of Paris Brick) had the lowest surface hardness, measuring Shore D 58. POP is softer and more brittle compared to resin-based composites and has a porous nature that predisposes it to higher surface wear and denting. Despite being reinforced with cement and shredded textiles, the surface is still less resistant to mechanical abrasion. Accordingly, CRB-03 is more appropriate for interior or decorative use where surface hardness is not a priority and physical contact is reduced.
\end{enumerate}
\begin{figure}[H]
	\begin{minipage}{1\textwidth}
		\centering
		\fbox{\includegraphics[height=10cm, width=1\linewidth]{Assets/29.png}}
		\caption{Surface Hardness Test Result}
	\end{minipage}
\end{figure}

% --------------- 4.1.4 --------------------%
\subsection{Flexural Strength Test Result}
\begin{table}[H]
	\renewcommand{\arraystretch}{2} % row height
	\setlength{\tabcolsep}{7pt} % column padding
	\begin{tabular}{
		|>{\centering\arraybackslash}m{2cm}|
		>{\centering\arraybackslash}m{2cm}|
		>{\centering\arraybackslash}m{2cm}|
		>{\centering\arraybackslash}m{2cm}|
		>{\centering\arraybackslash}m{2cm}|
		>{\centering\arraybackslash}m{2cm}|
		}
		\hline
		\rowcolor{gray!20}
		Sample Code. & Maximum Applied Load, P (kN) & Span Length, L (mm)  & Brick Width, b (mm)  & Depth, d (mm)       & Flexural Strength, R (MPa) \\ \hline
		CRB-01       & 1.25                         & \multirow{3}{*}{150} & \multirow{3}{*}{100} & \multirow{3}{*}{27} & 3.86                       \\
		\cline{1-2} \cline{6-6}
		CRB-02       & 1.15                         &                      &                      &                     & 3.55                       \\
		\cline{1-2} \cline{6-6}
		CRB-02       & 0.80                         &                      &                      &                     & 2.47                       \\
		\cline{1-2} \cline{6-6}
		\hline
	\end{tabular}
	\caption{Flexural Strength Test Result}
	\label{tab:placeholder}
\end{table}

\begin{figure}[H]
	\begin{minipage}{1\textwidth}
		\centering
		\fbox{\includegraphics[height=10cm, width=1\linewidth]{Assets/30.png}}
		\caption{Flexural Strength Test Result}
	\end{minipage}
\end{figure}

\begin{enumerate}
	\item CRB-01 (Epoxy Brick) was found to have a flexural strength of 3.86 MPa, demonstrating a high resistance to bending and cracking when under load. The good bonding of epoxy with shredded textiles enables stress to be efficiently transferred throughout the composite, enhancing the material's resistance to failure under the application of flexural forces. This qualifies CRB-01 for use in applications where bending stresses are prevalent, e.g., in thin wall panels, partition boards, or structural overlays, where surface continuity and durability are essential.
	\item CRB-02 (Polyester Brick) had a marginally lesser flexural strength of 3.55 MPa, which is still indicative of good performance. Polyester resin, although less rigid compared to epoxy, has superior ductility, which can be useful for resisting the propagation of cracks when subjected to bending. The presence of flexibility coupled with the reinforcement by shredded textiles enables the brick to absorb the bending loads without catastrophic fracture. The flexural capacity of CRB-02 recommends its application in lightweight structural elements or any surface area subjected to bending stress either during installation or in service.
	\item CRB-03 (Plaster of Paris Brick) showed the lowest flexural strength at 2.47 MPa, which is in line with the brittle nature of gypsum-based products. Despite the fact that the addition of shredded textiles and cement marginally increases the resistance of the brick to cracking, POP does not possess the structural toughness of resin-based systems inherently. Therefore, CRB-03 is more suitable for non-load-bearing uses with minimal flexural stress, e.g., decorative wall features or interior finishing items.
\end{enumerate}

% --------------- 4.1.5 --------------------%
\subsection{Water Absorption Test Result}
\begin{table}[H]
	\renewcommand{\arraystretch}{2} % row height
	\setlength{\tabcolsep}{10pt} % column padding
	\begin{tabular}{
		|>{\centering\arraybackslash}m{3cm}|
		>{\centering\arraybackslash}m{3cm}|
		>{\centering\arraybackslash}m{3cm}|
		>{\centering\arraybackslash}m{3cm}|
		}
		\hline
		\rowcolor{gray!20}
		Sample Code. & Weight before Submersion, W$_1$ (gm) & Weight after Submersion, W$_2$ (gm) & Water Absorption\% \\ \hline
		CRB-01       & 500                                  & 511.55                              & 2.31               \\ \hline
		CRB-02       & 560                                  & 581.78                              & 3.89               \\ \hline
		CRB-02       & 492                                  & 540.12                              & 9.78               \\ \hline
	\end{tabular}
	\caption{Water Absorption Test Result}
	\label{tab:placeholder}
\end{table}

\begin{figure}[H]
	\begin{minipage}{1\textwidth}
		\centering
		\fbox{\includegraphics[height=10cm, width=1\linewidth]{Assets/31.png}}
		\caption{Water Absorption Test Result}
	\end{minipage}
\end{figure}

\begin{enumerate}
	\item CRB-01 (Epoxy Brick) showed a water absorption rate of 2.31\%, which demonstrates the high resistance of epoxy resin to the penetration of water. Epoxy's tight, crosslinked network reduces the entry of water even in the presence of shredded textiles. This low absorption qualifies CRB-01 as a trusted material to be used in humid conditions or where water resistance is paramount.
	\item CRB-02 (Polyester Brick) had a marginally greater water absorption of 3.89\%. Although unsaturated polyester resin tends to be resistant to water, it is more permeable than epoxy, particularly when in composite form. The addition of shredded textiles and possible voids within the matrix are the cause of this modest degree of absorbency. CRB-02 remains appropriate for the majority of interior or semi-protected building uses but might need surface sealing in areas subjected to high moisture.
	\item CRB-03 (Plaster of Paris Brick) recorded the highest water absorption at 9.78\%, in line with the inherent porous and hygroscopic characteristics of gypsum-based products. Despite the incorporation of cement and shredded textiles, POP is prone to absorbing water easily, which can compromise its dimensional stability and durability in the long run. As it is, CRB-03 would be most suitable for use in dry interior areas or sheltered architectural elements where moisture exposure is kept to a bare minimum.
\end{enumerate}
\section{Results Overview}

\begin{table}[H]
	\renewcommand{\arraystretch}{2} % row height
	\setlength{\tabcolsep}{6pt} % column padding
	\begin{tabular}{
		|>{\centering\arraybackslash}m{2cm}|
		>{\centering\arraybackslash}m{2cm}|
		>{\centering\arraybackslash}m{2cm}|
		>{\centering\arraybackslash}m{2cm}|
		>{\centering\arraybackslash}m{2cm}|
		>{\centering\arraybackslash}m{2cm}|
		}
		\hline
		\rowcolor{gray!20}
		Sample Code & Compressive Strength (N/mm$^2$ or MPa) & Density (kg/m$^3$) & Surface Hardness (Shore D) & Flexural Strength (N/mm$^2$ or MPa) & Water Absorbency (\%) \\ \hline
		CRB-01      & 5.50                                   & 925.92             & 78                         & 3.86                                & 2.31                  \\ \hline
		CRB-02      & 6.50                                   & 1037.04            & 72                         & 3.55                                & 3.89                  \\ \hline
		CRB-03      & 3.50                                   & 907.41             & 58                         & 2.47                                & 9.78                  \\ \hline
	\end{tabular}
	\caption{Results Overview  }
	\label{tab:placeholder}
\end{table}
\section{Performance Comparison}

\renewcommand{\arraystretch}{2} % row height
\setlength{\tabcolsep}{10pt} % column padding
\begin{longtable}{
	|>{\centering\arraybackslash}m{3cm}|
	>{\centering\arraybackslash}m{3cm}|
	>{\centering\arraybackslash}m{3cm}|
	>{\centering\arraybackslash}m{3cm}|
	}
	\hline
	\rowcolor{gray!20}
	Parameter            & Epoxy Composite & PET Composite & POP Composite \\ \hline
	Compressive Strength & Moderate        & High          & Low           \\ \hline
	Density              & Moderate        & High          & Low           \\ \hline
	Surface Hardness     & High            & Moderate      & Low           \\ \hline
	Flexural Strength    & High            & Moderate      & Low           \\ \hline
	Water Absorption     & Low             & Moderate      & High          \\ \hline
	\caption{Performance Comparison  }
	\label{tab:placeholder}
\end{longtable}
\section{Comparison with Traditional Fired Clay Brick (FCB)}

\renewcommand{\arraystretch}{2} % row height
\setlength{\tabcolsep}{7pt} % column padding
\begin{longtable}{
	|>{\centering\arraybackslash}m{2.5cm}|
	>{\centering\arraybackslash}m{2.5cm}|
	>{\centering\arraybackslash}m{2.5cm}|
	>{\centering\arraybackslash}m{2.5cm}|
	>{\centering\arraybackslash}m{2.5cm}|
	}
	\hline
	\rowcolor{gray!20}
	Property                   & CRB-01 (Epoxy)                & CRB-02 (Polyester)             & CRB-03 (POP)                    & Traditional Fired Clay Brick (FCB)      \\ \hline
	Compressive Strength (MPa) & 5.50                          & 6.50                           & 3.50                            & 12.50                                   \\ \hline
	Density (kg/m$^3$)         & 925.92                        & 1037.04                        & 907.41                          & 1750                                    \\ \hline
	Water Absorption (\%)      & 2.31                          & 3.89                           & 9.78                            & 17.50                                   \\ \hline
	Surface Hardness (Shore D) & 78                            & 72                             & 58                              & 85                                      \\ \hline
	Flexural Strength (MPa)    & 3.86                          & 3.55                           & 2.47                            & 4.10                                    \\ \hline
	Durability                 & High chemical resistance      & Good outdoor durability        & Limited (prone to water damage) & Prone to erosion, weathering            \\ \hline
	Eco friendliness           & Uses textile waste            & Uses textile waste             & Uses textile waste              & Energy intensive, high CO$_2$ emissions \\ \hline
	Construction Use           & Non-load earing walls, panels & Partition and light load walls & Decorative, false walls         & Load-bearing structural walls           \\ \hline
	\caption{Comparison with Traditional Clay Brick}
	\label{tab:placeholder}
\end{longtable}

\begin{figure}[H]
	\begin{minipage}{1\textwidth}
		\centering
		\fbox{\includegraphics[height=10cm, width=1\linewidth]{Assets/32.png}}
		\caption{Compressive Strength Comparison with Traditional Brick }
	\end{minipage}
\end{figure}

The compressive strength comparison overall indicates that conventional red brick possesses the highest value of 12.5 MPa, whereas the recycled textile-reinforced bricks CRB-02 (Polyester), CRB-01 (Epoxy), and CRB-03 (POP) yielded 6.5 MPa, 5.5 MPa, and 3.5 MPa, respectively. Even though these are inferior to conventional clay brick value, the recycled composite bricks possess some important advantages. They utilize shredded textile waste, aiding in solid waste minimization, reduced carbon footprint, and sustainable resources management. Moreover, these can be designed to be lighter in weight, which could help in the reduction of transportation cost and structural load, and the properties can be tailored based on the resin and fiber combination. Among the composites, the polyester-based CRB-02 was found to perform the best, and hence it has the potential to be used in non-load-bearing or partition wall applications. For improving their structural feasibility, improvement in material formulation - such as better fiber-matrix bonding and hybrid reinforcement is required. However, the environmental and economic advantages make textile-reinforced brick a promising alternative in green construction practices.

\begin{figure}[H]
	\begin{minipage}{1\textwidth}
		\centering
		\fbox{\includegraphics[height=10cm, width=1\linewidth]{Assets/33.png}}
		\caption{Density Comparison with Traditional Brick }
	\end{minipage}
\end{figure}

The bar graph displays the density values (kg/m$^3$) of four types of bricks: CRB-01 (Epoxy), CRB-02 (Polyester), CRB-03 (POP), and Traditional Clay Brick. Among the samples, traditional clay brick has the highest density at 1750.00 kg/m$^3$, being the heaviest and densest material. Conversely, the textile-reinforced composite bricks have very low densities, which can be useful in applications where light materials are required. CRB-02 (Polyester) is the most dense among the composites with a value of 1037.04 kg/m$^3$, followed by CRB-01 (Epoxy) with a value of 925.92 kg/m$^3$, and CRB-03 (POP) with 907.41 kg/m$^3$. The lightest material is CRB-03 and may provide benefits to structures that require lower dead loads. Whereas traditional bricks give higher mass and potentially more stability under load-bearing applications, the low density of composite bricks means easier handling, transportation, and possible thermal insulation advantages. Therefore, from the standpoint of weight efficiency, CRB-02 offers a good compromise between strength and density, while CRB-03 is better in terms of being the lightest, rendering these composite alternatives more appropriate for lightweight or modular building systems.

\begin{figure}[H]
	\begin{minipage}{1\textwidth}
		\centering
		\fbox{\includegraphics[height=10cm, width=1\linewidth]{Assets/34.jpg}}
		\caption{ Hardness Comparison with Traditional Brick}
	\end{minipage}
\end{figure}

The bar chart illustrating the Hardness (Shore D) of different materials provides valuable insight into their surface durability and resistance to deformation, which are critical for structural and wear-intensive applications. Traditional clay bricks again show the highest hardness value at 85 Shore D, reinforcing their established reputation for robustness and long-term durability in construction. Among the composite alternatives, Epoxy-based bricks (CRB-01) exhibit a high hardness of 78 Shore D, suggesting excellent resistance to surface wear, scratches, and minor impacts, making them a strong candidate for replacing traditional materials in both load-bearing and surface-exposed applications. Polyester-based bricks (CRB-02) follow with a moderate hardness of 72 Shore D, indicating acceptable durability but with slightly reduced performance under abrasive or impact conditions. POP-based bricks (CRB-03) record the lowest hardness at 58 Shore D, highlighting their soft and brittle nature, which limits their practical use to decorative or low-impact applications. Overall, the data strongly supports the use of epoxy as the most effective binder among the tested alternatives when hardness and surface resilience are key performance criteria in composite brick manufacturing.

\begin{figure}[H]
	\begin{minipage}{1\textwidth}
		\centering
		\fbox{\includegraphics[height=10cm, width=1\linewidth]{Assets/35.jpg}}
		\caption{Flexural Strength Comparison with Traditional Brick }
	\end{minipage}
\end{figure}

The bar chart comparing the Flexural Strength (MPa) of various materials reveals critical insights into the mechanical performance of each binder used in composite brick production. Traditional clay bricks exhibit the highest flexural strength at 4.10 MPa, indicating their superior load-bearing capability, which aligns with their long-standing use in construction. Among the composite alternatives, Epoxy-based bricks (CRB-01) demonstrate the highest strength at 3.86 MPa, closely rivaling traditional clay, and suggesting that epoxy offers excellent bonding and structural integrity within the composite matrix. Polyester-based bricks (CRB-02) follow with a slightly lower strength of 3.55 MPa, still within a usable range but indicating slightly less rigidity and bonding efficiency compared to epoxy. In contrast, Plaster of Paris (POP)-based bricks (CRB-03) show significantly lower flexural strength at 2.47 MPa, which suggests limited structural performance and suitability only for non-load-bearing or decorative applications. These results suggest that epoxy resin is the most viable alternative binder for achieving high structural performance in eco-friendly composite bricks, closely emulating or even surpassing traditional clay when optimized further.

\begin{figure}[H]
	\begin{minipage}{1\textwidth}
		\centering
		\fbox{\includegraphics[height=10cm, width=1\linewidth]{Assets/36.png}}
		\caption{Water Absorption Comparison with Traditional Brick }
	\end{minipage}
\end{figure}

The results of the water absorption test indicate considerable variation amongst the four types of bricks. CRB-01 (Epoxy) gave the lowest water absorption of 2.31\%, which is the best amongst all the samples, making the material the most resistant to water. This is due to the fact that the dense and impermeable nature of epoxy's polymer network effectively keeps moisture out. Next, CRB-02 (Polyester) presented a slightly higher but still low absorption of 3.89\%, proving resistant to water and the second-best performing material. CRB-03 (POP) presented a considerably higher absorption of 9.78\%, with only moderate resistance to water but still an improvement compared to conventional clay brick. Traditional clay brick, on the other hand, gave the highest absorption of 17.50\%, demonstrating its porous nature and susceptibility to moistureinduced damage. Of the four, CRB-01 is by far the most resistant to water, followed by CRB-02, while CRB-03, despite being less efficient, still outperforms conventional clay brick. The findings indicate that the recycled textile-reinforced composite bricks, specifically the epoxy or polyester resin-based ones, provide enhanced protection from water penetration, making them more durable and suitable for application in damp or moisture-susceptible areas.
\section{Application and End Use}

% --------------- 4.5.1 --------------------%
\subsection{Epoxy Composite Brick}
Due to their moderate compressive strength, they are well-suited for moisture-resistant applications. Their durability and resilience make them ideal for decorative features and eco-friendly construction. \\

\noindent Application Areas:
\begin{enumerate}
	\item Non-load bearing walls.
	\item Industrial flooring \& wall-base.
	\item Protective enclosure.
	\item Moisture prone Areas (Kitchens, Bathrooms).
	\item External facade panels.
	\item Decorative features (Accent Walls, Panels).
	\item Sustainable construction (Green building).
\end{enumerate}

% --------------- 4.5.2 --------------------%
\subsection{Polyester Composite Brick}
Polyester bricks, which are created by mixing shredded textile waste with polyester resin and hardeners, which are more powerful than epoxy bricks. Their superior compressive strength them appropriate for load-bearing uses. Polyester bricks are especially suitable for prefabricated and modular home solutions. \\

\noindent Application Areas:
\begin{enumerate}
	\item Non-load bearing / Partition walls.
	\item Boundary \& Compound walls.
	\item Facade cladding.
	\item Pavement \& footpath construction.
	\item Prefabricated and modular housing.
	\item Sustainable construction (Eco-friendly projects).
\end{enumerate}

% --------------- 4.5.3 --------------------%
\subsection{Plaster of Paris (POP) Composite Brick}
Plaster of Paris bricks are non-load-bearing, lightweight materials that are manufactured by blending shredded textile waste with POP and cement. Since they possess poor compressive strength, they are used mostly for decorative and temporary purposes in construction. \\

\noindent Application Areas:
\begin{enumerate}
	\item Patterned/Decorative interior walls.
	\item Decorative false ceilings.
	\item Exhibition booths, Display stands.
	\item Low height edging, border in gardens or parks.
	\item Carving for ornamental detailing.
	\item Sustainable construction.
\end{enumerate}