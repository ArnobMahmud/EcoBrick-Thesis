\chapter{Conclusion}
The research explores a novel, eco-friendly, and structurally viable way of integrating post-consumer and industrial textile waste in composite brick production. The method was conceived against the backdrop of Bangladesh's urgency to find sustainable answers to the mounting textile waste crisis and the construction industry's quest for alternative, affordable construction materials.\\

\noindent A main objective of this project was the incorporation of chopped textile waste (100gm) consisting mainly of cotton, polyester, denim, and mixed woven/knitted fabrics into various resin matrices, such as epoxy resin (300 ml with 150 ml hardener), polyester resin (300 ml with 4.5 – 6 ml MEKP and 1.5 - 3ml cobalt accelerator), and plaster of paris (POP, 400 gm with 200 ml water). These binder systems were chosen due to their diverse mechanical properties, availability, cost, and compatibility with textile waste, thereby providing a wide range for comparative performance. \\

\noindent Composite bricks were made in a standard mold (200mm $\times$ 100mm $\times$60mm) and tested as per applicable ASTM standards. The mechanical and physical properties that were evaluated were compressive strength, flexural strength, density, surface hardness, and water absorption. Statistical analysis of shredded textile fiber length - mean (18.9 mm), standard deviation (3.75), and coefficient of variation (19.84\%) - also gave an indication of the consistency and distribution, further adding to the credibility of raw materials utilized.\\

\noindent The composite bricks with epoxy resin proved to be the most potential type with moderate compressive strength (5.50 MPa), superior flexural resistance (8.64 MPa), and high surface hardness (78 Shore D). All these characteristics reveal the brick's potential to resist structural loads as well as environmental exposure, and thus it is found to be very useful for interior and exterior construction applications. [27] \\

\noindent Composites based on the polyester resin, while being somewhat weaker and more prone to water uptake than epoxy brick, still exhibited high mechanical behavior (6.50 MPa compressive strength) and economic viability. Their relatively quicker cure, along with better handling, positions them for large-volume, low-cost production-perfect for developing countries where material cost and resource efficiency take precedence. \\

\noindent Conversely, composite bricks based on POP showed poor mechanical strength (3.50 MPa) but high - water absorption (9.78\%) due to its porous nature. However, they do find merit in temporary, decorative, or non-load-bearing uses, where aesthetic incorporation of recycled textiles may be prioritized over structural performance. \\

\noindent From an environmental and sustainability point of view, this study is in line with the circular economy concept by introducing a scalable and replicable framework for waste-to-resource conversion. As one of the highest generators of textile waste in the world, Bangladesh can greatly gain from such innovation. Not only does it divert huge amounts of textile waste from land filling and incineration, but it also helps to lessen the construction sector's ecological footprint. \\

\noindent In summary, this project effectively illustrates the technical viability, economic viability, and environmental value of the use of chopped textile waste as reinforcement in resin- and gypsum-based composite bricks. It creates a new avenue for cross disciplinary cooperation among the textile, construction, and environmental industries. Long-term durability tests, field performance assessments, resin-to-fiber ratio optimization, bending strength, and investigation of biodegradable or bio-based resins should be considered in future studies to further promote sustainability and eco performance.