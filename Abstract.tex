\begin{abstract}
Recycling textile waste into building materials offers a double benefit for waste management as well as for green building. This study investigates the possibility of using recycled garment textile waste as reinforcement in the manufacturing of bricks with specific reference to Bangladesh where ready-made garment production generates approximately 577,000 metric tons of waste annually and the traditional clay brick kilns contribute up to 58\% of Dhaka's air pollution. The research entails large literature review, experimental development of textile reinforced composite bricks, and evaluation of their mechanical, economic, and environmental performance. Real experimental results from the latest research are brought together: bricks of 18\% waste fabric by weight and polymer-based binders were of compressive strengths in the range of approximately 3 - 7 MPa (435 - 1015 psi) with 60\% lower weight. Low density and fiber content improved thermal insulation according to earlier studies, in which cotton/textile-infused bricks have been found to comply with ASTM standards of strength and conductivity. An ecological view presents immense advantage: the proposed "eco-bricks" conserve firing (hence no kiln emissions), utilize textile waste that otherwise would pollute rivers and landfills, and reduce consumption of virgin clay soil. Cost analysis determines that a potential 20 - 40\% cost reduction per brick over fired clay bricks can make safe housing affordable for low-income communities. Policy recommendations include green brick manufacturing incentives, recycling streams for waste integrated into construction supply chains, and building code reforms enabling new composite brick solutions. The research confirms that composite bricks composed of recycled textile waste are durable, light-weight, and thermally efficient building blocks that embody constructive ingenuity towards sustainable development. The research contributes a scalable model for developing economies to turn industrial waste into value-added construction materials, addressing environmental pollution as well as housing needs at the same time. 
\end{abstract}