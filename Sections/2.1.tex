\section{Textile Waste: Global Trends and Bangladesh Scenario}

Worldwide, the textile and fashion industry are one of the most resource-consuming sectors that generate more than 92 million tons of waste each year. In Bangladesh, the RMG sector, although earning about 84\% of the country's export revenue, produces about 400,000 tons of solid textile wastes yearly. These cotton, polyester, denim, and mixed woven and knitted fabric-based wastes are generally disposed of in landfills or incinerated without any formal recycling system, which contributes to serious environmental issues like groundwater pollution and greenhouse gas emissions. \\

\noindent Research (by Hasan et al., 2020 and Rahman and Ahsan, 2019) shows that most factories in Dhaka, Narayanganj, and Chattogram lack systematic textile waste management protocols, though up to 25\% of production material is wasted per garment unit.