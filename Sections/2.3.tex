\section{Resin-Bonded Composites in Construction}
The employment of resin-based systems as binders has gained prominence in the formulation of new construction materials. Some of the most common ones used are polyester resin, epoxy resin, and plaster of Paris (POP) with each of them possessing different mechanical and chemical characteristics for composite development. \\

\noindent Polyester resin, due to its low price, simple processing, and good mechanical properties, is a popular choice. It has seen widespread application in polymer concrete, fiber reinforced panels, and composite tiles. Polyester resin, when blended with shredded textile waste, especially synthetic shredded textiles like polyester and natural shredded textiles like cotton, offers good adhesion, forming a stiff matrix that increases compressive strength and dimensional stability. Epoxy resin, being costlier than polyester resin, provides higher bonding strength, minimal shrinkage, and very good chemical resistance. This makes it highly suitable for structural applications where the requirement is high durability and resistance to moisture. Research has established that epoxy-bound textile composites have better mechanical integrity and service life when subjected to environmental stress than with other binding systems. Studies suggest that epoxy-based bricks show higher compressive and flexural strength compared to polyester-based ones due to better matrix-fiber bonding (Hussain et al., 2021). Plaster of Paris (POP), which mainly consists of calcium sulfate hemihydrate, is extensively utilized in low-load applications owing to its quick setting and good finish. As a partial binder or filler in composite bricks, POP has the potential to enhance thermal insulation and surface finish. Nevertheless, it needs careful formulation to prevent brittleness and water solubility. Research indicates that combining POP with additives or shredded textiles can enhance its performance and durability in composite bricks (Ahmed et al., 2018). \\

\noindent In this study, all the three matrices - polyester resin, epoxy resin, and POP were used in different proportions to balance the strength, curing time, and economy of the composite bricks. Their compatibility with different kinds of textile waste (cotton, polyester, denim, woven, and knitted fabrics) was investigated systematically to arrive at the most suitable formulation for structural and environmental sustainability. \\

\noindent Several studies have focused on evaluating compressive strength, flexural strength, surface hardness, thermal conductivity, and water absorption of composite bricks. The ASTM standards such as:

\begin{enumerate}
	\item ASTM C67 for compressive strength,
	\item ASTM C293 for flexural strength,
	\item ASTM C642 for density,
	\item ASTM D2240 for surface hardness, and
	\item ASTM D570 for water absorption are commonly adopted for these evaluations.
\end{enumerate}
