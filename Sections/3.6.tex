\section{Test Methodologies}
ach test was conducted following ASTM standards, ensuring reliability and
repeatability. The following procedures were adopted:

% --------------- 3.6.1 --------------------%
\subsection{Compressive Strength Test}
\begin{enumerate}
	\item Standard: ASTM C67.
	\item Equipment: E159-01D Concrete Compression Machine - MATEST 250kN.
	\item Procedure:
	      \begin{itemize}
		      \item Bricks were placed horizontally on the compression plate.
		      \item Load was applied gradually at a rate of 14MPa/min until the sample failed.
		      \item Compressive strength was calculated as:
		            \[
			            \text{Compressive Strength} =
			            \frac{\textit{Load at Failure (N)}}{\textit{Cross sectional area (mm$^2$)}}
		            \]
	      \end{itemize}
\end{enumerate}

% --------------- 3.6.2 --------------------%
\subsection{Density Test}
\begin{enumerate}
	\item Standard: ASTM C642.
	\item Procedure:
	      \begin{itemize}
		      \item Dry mass measured after oven drying.
		      \item Volume calculated using:
		            \[
			            \text{Volume} = \text{Length} \times \text{Width} \times \text{Height} \; (\text{m}^3)
		            \]
		      \item Compressive strength was calculated as:
		            \[
			            \text{Compressive Strength} =
			            \frac{\textit{Dry Mass (Kg)}}{\textit{Volume (mm$^3$)}}
		            \]
	      \end{itemize}
\end{enumerate}

\begin{figure}[H]
	\centering
	\begin{minipage}{0.48\textwidth}
		\centering
		\includegraphics[height=10cm, width=1\linewidth, keepaspectratio]{Assets/20.jpg}
		\caption{Compressive Strength Testing Machine  }
	\end{minipage}\hfill
	\begin{minipage}{0.48\textwidth}
		\centering
		\includegraphics[height=10cm, width=1\linewidth, keepaspectratio]{Assets/21.jpg}
		\caption{Flexural Strength Testing Machine}
	\end{minipage}

	\centering
	\begin{minipage}{0.48\textwidth}
		\centering
		\vspace{1cm}
		\includegraphics[height=8cm, width=\linewidth]{Assets/22.jpg}
		\caption{Weight Balance}
	\end{minipage} \hfill
	\begin{minipage}{0.48\textwidth}
		\centering
		\vspace{1cm}
		\includegraphics[height=8cm, width=1\linewidth, keepaspectratio]{Assets/23.jpg}
		\caption{Water Drum}
	\end{minipage}

\end{figure}


\begin{figure}[H]
	\centering
	\begin{minipage}{0.48\textwidth}
		\centering
		\includegraphics[height=8cm, width=1\linewidth]{Assets/24.jpg}
		\caption{Oven}
	\end{minipage}\hfill
	\begin{minipage}{0.48\textwidth}
		\centering
		\includegraphics[height=8cm, width=1\linewidth]{Assets/25.jpg}
		\caption{Durometer}
	\end{minipage}

	\begin{minipage}{1\textwidth}
		\centering
		\vspace{1cm}
		\includegraphics[height=7cm, width=1\linewidth]{Assets/26.jpg}
		\caption{Meter Scale and Vernier Scale}
	\end{minipage}
\end{figure}

% --------------- 3.6.3 --------------------%
\subsection{Surface Hardness Test}
\begin{enumerate}
	\item Standard: ASTM D2240.
	\item Procedure:
	      \begin{itemize}
		      \item Durometer (Shore D) is used.
		      \item Applied to 5 random flat surface points per brick.
		      \item Average hardness value recorded.
	      \end{itemize}
\end{enumerate}

% --------------- 3.6.4 --------------------%
\subsection{Flexural/Bending Strength Test}
\begin{enumerate}
	\item Standard: ASTM C293.
	\item Equipment: Servo-Plus Evolution Flexural Tester
	\item Procedure:
	      \begin{itemize}
		      \item Bricks placed on two supports, span length 150 mm.
		      \item Load applied at midpoint until crack/failure.
		      \item Bending strength:
		            \[
			            \text{Bending Strength, R} =
			            \frac{\textit{3PL}}{\textit{2bd$^2$}} \text{MPa}
		            \]
	      \end{itemize}
\end{enumerate}

% --------------- 3.6.5 --------------------%
\subsection{Water Absorption Test}
\begin{enumerate}
	\item Standard: ASTM D570.
	\item Procedure:
	      \begin{itemize}
		      \item Dry the test specimens (bricks) in an oven at 50°C to 60°C for at least 3 hours.
		      \item Cool in a desiccator and weigh the dry sample (W${_1}$).
		      \item Immerse the bricks in distilled water at 25°C for 24 or 48 hours.
		      \item Remove specimens, wipe the surface dry.
		      \item Weigh the wet sample immediately (W${_2}$).
	      \end{itemize}
	\item  Calculation:
	      \[
		      \text{Water Absorption\%} =
		      \frac{\textit{W${_2}$ - W${_1}$}}{\textit{W${_1}$}} \times 100\%
	      \]
\end{enumerate}