\section{Composite Materials from Recycled Textiles}

Textile waste, particularly from the garment industry, poses a serious environmental challenge due to its slow degradation and volume. Researchers have shown that chopped textile can be incorporated into construction materials to enhance properties such as flexibility, insulation, and toughness (Yin et al., 2021). Waste textile are increasingly considered viable reinforcements in polymer or gypsum matrices for brick and panel production, owing to their fibrous nature and energy-absorbing capacity (Miah et al., 2020). \\

\noindent A composite textile in resin matrix is a fiber-reinforced polymer (FRP) material, where shredded textile (reinforcement) are embedded in a resin matrix (binder), resulting in a durable and lightweight construction material. Several international studies have examined the mechanical and thermal applications of recycled textiles in construction. Shredded textiles such as cotton and polyester have been successfully used in insulation boards, low-strength concrete, geo-textiles, and polymer-based composites. Shredded cotton improves the ductility of cement composites, while synthetic shredded textiles enhance dimensional stability and moisture resistance (Park et al., 2018). Studies have highlighted that composite bricks made from alternative binders can perform competitively in terms of mechanical and thermal properties when designed appropriately (Ghosh et al., 2020; Zhang et al., 2019). 