\section{Brick Fabrication Method}

% --------------- 3.4.1 --------------------%
\subsection{Epoxy Resin - Textile Composite Formulation}

\noindent\underline{Step -1: Material Collection}
\begin{itemize}[leftmargin=1.5cm]
	\item Gather shredded textile waste (cotton, polyester, denim, woven, knitted).
	\item Measure 100 gm of shredded textiles (size approx. 15 - 30 mm).
\end{itemize}

\noindent\underline{Step -2: Resin Mixing }
\begin{itemize}[leftmargin=1.5cm]
	\item Measure 300 mL of epoxy resin (Part A).
	\item Add 150 mL of hardener (Part B) to maintain a 2:1 ratio.
	\item Mix the resin and hardener thoroughly for 2 - 3 minutes into a beaker.
\end{itemize}

\noindent\underline{Step -3: Chopped Waste Addition}
\begin{itemize}[leftmargin=1.5cm]
	\item Gradually add the 100 gm of chopped textile fabrics into the resin mixture.
	\item Stir continuously to ensure homogeneous dispersion.
	\item Avoid air bubbles by mixing slowly and thoroughly.
\end{itemize}

\noindent\underline{Step -4: Mold Preparation}
\begin{itemize}[leftmargin=1.5cm]
	\item Use a wooden mold (200 mm $\times$ 100 mm $\times$ 60 mm).
	\item Clean and, if necessary, apply mold release to prevent sticking.
\end{itemize}

\noindent\underline{Step -5: Casting}
\begin{itemize}[leftmargin=1.5cm]
	\item Pour the prepared composite mixture into the mold evenly.
	\item Compact gently to remove air pockets and ensure proper filling.
	\item Apply pressure for perfect shape achievement.
\end{itemize}

\noindent\underline{Step -6: Curing}
\begin{itemize}[leftmargin=1.5cm]
	\item Allow the mold to cure at room temperature for 2 - 4 hours.
	\item Avoid disturbance during curing to ensure dimensional stability.
\end{itemize}

\noindent\underline{Step -7: De-molding}
\begin{itemize}[leftmargin=1.5cm]
	\item Once fully cured, carefully remove the composite brick from the mold.
	\item Inspect the surface and edges for any visible defects.
\end{itemize}

\noindent\underline{Step -8: Labeling}
\begin{itemize}[leftmargin=1.5cm]
	\item Assign a unique code for brick sample (CRB-01).
	\item Store the brick in a dry, dust-free environment before testing.
\end{itemize}

\begin{table}[H]
	\centering
	\renewcommand{\arraystretch}{2} % row height
	\setlength{\tabcolsep}{10pt} % column padding
	\begin{tabular}{|>{\centering\arraybackslash}m{4cm}|>{\centering\arraybackslash}m{4cm}|>{\centering\arraybackslash}m{4cm}|}
		\hline
		\rowcolor{gray!20}
		Sample Code             & Components            & Amount \\
		\hline
		\multirow{3}{*}{CRB-01} & Chopped Textile Waste & 100 gm \\
		\cline{2-3}
		                        & Epoxy Resin           & 300 mL \\
		\cline{2-3}
		                        & Hardener              & 150 mL \\
		\cline{2-3}
		\hline
	\end{tabular}
	\caption{Epoxy Resin -Textile Composite Recipe }
\end{table}

% --------------- 3.4.2 --------------------%
\subsection{Polyester Resin - Textile Composite Formulation}

\noindent\underline{Step -1: Material Collection}
\begin{itemize}[leftmargin=1.5cm]
	\item Shredded textile waste (cotton, polyester, denim, woven/knitted).
	\item Measure 100 gm of chopped shredded textiles (approx. 15 - 30 mm length).
\end{itemize}

\noindent\underline{Step -2: Resin Preparation}
\begin{itemize}[leftmargin=1.5cm]
	\item Measure 300 mL of polyester (PET) resin.
\end{itemize}

\noindent\underline{Step -3: Catalyst and Accelerator Addition }
\begin{itemize}[leftmargin=1.5cm]
	\item Add 1.5 - 2\% MEKP (4.5 - 6 mL for 300 mL resin) as the hardening catalyst.
	\item Add 0.5 - 1\% cobalt accelerator (1.5 - 3 mL) to initiate and control curing.
	\item Mix well under ventilated conditions and with appropriate safety precautions.
\end{itemize}

\noindent\underline{Step -4: Chopped Waste Addition }
\begin{itemize}
	\item Slowly add the 100 gm chopped textile into the resin mixture.
	\item Mix thoroughly to ensure even fiber distribution and prevent clumping.
\end{itemize}

\noindent\underline{Step -5: Mold Preparation }
\begin{itemize}[leftmargin=1.5cm]
	\item Use the same wooden mold (200 mm $\times$ 100 mm $\times$ 60 mm).
	\item Clean the mold and apply a release agent if necessary.
\end{itemize}

\noindent\underline{Step -6: Casting}
\begin{itemize}[leftmargin=1.5cm]
	\item Pour the composite mix evenly into the mold.
	\item Lightly tap or vibrate the mold to release trapped air.
	\item Apply pressure for perfect shape achievement.
\end{itemize}

\noindent\underline{Step -7: Curing }
\begin{itemize}[leftmargin=1.5cm]
	\item Leave the mixture to cure at room temperature for 2 - 4 hours.
	\item Ensure a dust-free, undisturbed environment for optimal setting.
\end{itemize}

\noindent\underline{Step -8: De-molding}
\begin{itemize}[leftmargin=1.5cm]
	\item Carefully remove the cured composite from the mold.
	\item Check for surface integrity and consistency.
\end{itemize}

\noindent\underline{Step -9: Labeling }
\begin{itemize}[leftmargin=1.5cm]
	\item Assign a unique code for brick sample (CRB-02).
	\item Store the brick in a dry, dust-free environment before testing.
\end{itemize}

\begin{table}[H]
	\centering
	\renewcommand{\arraystretch}{2} % row height
	\setlength{\tabcolsep}{10pt} % column padding
	\begin{tabular}{|>{\centering\arraybackslash}m{4cm}|>{\centering\arraybackslash}m{4cm}|>{\centering\arraybackslash}m{4cm}|}
		\hline
		\rowcolor{gray!20}
		Sample Code             & Components            & Amount   \\
		\hline

		\multirow{4}{*}{CRB-02} & Chopped Textile Waste & 100 gm   \\
		\cline{2-3}
		                        & PET Resin             & 300 mL   \\
		\cline{2-3}
		                        & MEKP                  & 4.5-6 mL \\
		\cline{2-3}
		                        & Cobalt                & 1.5-3 mL \\
		\cline{2-3}

		\hline
	\end{tabular}
	\caption{Polyester Resin -Textile Composite Recipe }
\end{table}

% --------------- 3.4.3 --------------------%
\subsection{Plaster of Paris (POP) - Textile Composite Formulation}

\noindent\underline{Step -1: Material Collection }
\begin{itemize}[leftmargin=1.5cm]
	\item Shredded textile waste (cotton, polyester, denim, woven/knitted).
	\item Measure 100 gm of chopped shredded textiles (approx. 15 - 30 mm length).
\end{itemize}

\noindent\underline{Step -2: POP Mixture Preparation}
\begin{itemize}[leftmargin=1.5cm]
	\item Measure 400 gm of plaster of paris (POP).
	\item Add 200 mL clean water gradually to the 400 gm POP, following an approximate 2:1 ratio (POP:water).
	\item Stir until a smooth, lump-free paste forms.
	\item Add 50 gm grey cement for better binding.
\end{itemize}

\noindent\underline{Step -3: Chopped Waste Addition }
\begin{itemize}[leftmargin=1.5cm]
	\item Slowly add the 100 gm chopped textile into the resin mixture.
	\item Mix thoroughly to ensure uniform fiber distribution.
	\item Work quickly, as POP sets within 10 - 15 minutes.
\end{itemize}

\noindent\underline{Step -4: Mold Preparation }
\begin{itemize}
	\item Use the same wooden mold (200 mm $\times$ 100 mm $\times$ 60 mm).
	\item Clean the mold and apply a release agent if necessary.
\end{itemize}

\noindent\underline{Step -5: Casting }
\begin{itemize}[leftmargin=1.5cm]
	\item Pour the composite mix evenly into the mold.
	\item Level the surface and gently tap the mold to eliminate air bubbles.
	\item Apply pressure for perfect shape achievement.
\end{itemize}

\noindent\underline{Step -6: Curing }
\begin{itemize}[leftmargin=1.5cm]
	\item Allow to set undisturbed for 30 - 60 minutes.
	\item Let the composite dry for 2 - 4 hours under shade or room conditions to avoid cracking.
\end{itemize}

\noindent\underline{Step -7: De-molding }
\begin{itemize}[leftmargin=1.5cm]
	\item Carefully remove the cured composite from the mold.
	\item Air-dry completely before testing or further processing.
\end{itemize}

\noindent\underline{Step -8: Labeling }
\begin{itemize}[leftmargin=1.5cm]
	\item Assign a unique code for brick sample (CRB-03).
	\item Store the brick in a dry, dust-free environment before testing.
\end{itemize}

\begin{table}[H]
	\centering
	\renewcommand{\arraystretch}{2} % row height
	\setlength{\tabcolsep}{10pt} % column padding
	\begin{tabular}{|>{\centering\arraybackslash}m{4cm}|>
		{\centering\arraybackslash}m{4cm}|>
		{\centering\arraybackslash}m{4cm}|}
		\hline
		\rowcolor{gray!20}
		Sample Code             & Components            & Amount \\
		\hline
		\multirow{4}{*}{CRB-03} & Chopped Textile Waste & 100 gm \\
		\cline{2-3}
		                        & POP                   & 400 gm \\
		\cline{2-3}
		                        & Water                 & 200 mL \\
		\cline{2-3}
		                        & Cement                & 50 gm  \\
		\cline{2-3}
		\hline
	\end{tabular}
	\caption{Plaster of Paris (POP) -Textile Composite Recipe }
\end{table}