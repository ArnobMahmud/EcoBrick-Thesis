\section{Research Limitations }
Although the research presents an innovative and environmentally compelling approach, several limitations may affect its scope and generalizability: 

\begin{enumerate}
\item Material Heterogeneity: Textile waste, particularly from mixed fiber sources such as denim, knitted, and woven fabrics, varies in texture, tensile strength, and dye content, which may influence consistency in composite formation. 

\item Resin Selection: The study is limited to specific types of resin (e.g., polyester, epoxy, or bio-based resins) due to availability and cost constraints. Resin toxicity and curing requirements may also pose environmental and safety considerations. 

\item Scale of Production: Fabrication is limited to a laboratory or pilot scale. Industrial-scale production, cost modeling, and long-term field testing remain outside the scope of this study. 

\item Time Constraints: Due to the academic calendar, the study does not include long-term environmental exposure or weathering tests, which are critical to validate outdoor performance. 

\item Regulatory Hurdles: Introducing a non-traditional material into mainstream construction will require alignment with local building codes and may encounter resistance from stakeholders unfamiliar with composite technologies. 
\end{enumerate}