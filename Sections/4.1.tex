\section{Test Result and Discussion}

% --------------- 4.1.1 --------------------%
\subsection{Compressive Strength Test Result}
\begin{table}[h!]
	\renewcommand{\arraystretch}{2} % row height
	\setlength{\tabcolsep}{10pt} % column padding
	\begin{tabular}{
		|>{\centering\arraybackslash}m{3cm}|
		>{\centering\arraybackslash}m{3cm}|
		>{\centering\arraybackslash}m{3cm}|
		>{\centering\arraybackslash}m{3cm}|
		}
		\hline
		\rowcolor{gray!20}
		Sample Code. & Load at Failure (kN) & Area under Load (mm$^2$) & Compressive Strength (N/(mm$^2$) or MPa) \\ \hline
		CRB-01       & 110                  & \multirow{3}{*}{20000}   & 5.50                                     \\
		\cline{1-2} \cline{4-4}
		CRB-02       & 130                  &                          & 6.50                                     \\
		\cline{1-2} \cline{4-4}
		CRB-03       & 70                   &                          & 3.50                                     \\
		\cline{1-2} \cline{4-4}
		\hline
	\end{tabular}
	\caption{Compressive Strength Test Result}
	\label{tab:placeholder}
\end{table}

\begin{figure}[H]
	\begin{minipage}{1\textwidth}
		\centering
		\fbox{\includegraphics[height=10cm, width=1\linewidth]{Assets/27.png}}
		\caption{Compressive Strength Test Result}
	\end{minipage}
\end{figure}

\begin{enumerate}
	\item CRB-01 (Epoxy Brick) attained a compressive strength of 5.50 MPa, which signifies good mechanical behavior and structural soundness. Epoxy resin is a rigid and highly cross linked matrix that strongly bonds with shredded textiles. The bonding action improves the inner structure of the brick, enabling it to sustain compressive loads nicely. The inherent brittleness of epoxy, though, can cause sudden failure at high stress in the presence of voids or micro cracks. Nevertheless, CRB-01 is still a good choice for light structural applications where strength and resistance to moisture are required.

	\item CRB-02 (Polyester Brick) had the greatest compressive strength among the three, at 6.50 MPa. This may be due to the fact that unsaturated polyester resin is a more flexible material that distributes stress more evenly and retains better adhesion with different textile reinforcements. The resin-shredded textile mix formed a dense, well consolidated matrix that could resist higher compressive forces. Consequently, CRB-02 is the most structurally sound of the test bricks and is highly suitable for application in lightweight structural members or partition walls.

	\item CRB-03 (Plaster of Paris Brick) had the lowest compressive strength at 3.50 MPa. Plaster of Paris, while being malleable and light, is inherently porous and brittle. Despite the incorporation of cement and shredded textiles, its compressive stress resistance capacity is still low. The shredded textiles can enhance crack resistance to some extent, but they cannot make up for the POP matrix's weakness. CRB-03 is thus most ideal for non-load-bearing purposes like decorative panels or interior partitions where mechanical strength is not important.
\end{enumerate}

% --------------- 4.1.2 --------------------%
\subsection{Density Test Result}
\begin{table}[h!]
	\renewcommand{\arraystretch}{2} % row height
	\setlength{\tabcolsep}{5.5pt} % column padding
	\begin{tabular}{
		|>{\centering\arraybackslash}m{1.715cm}|
		>{\centering\arraybackslash}m{1.715cm}|
		>{\centering\arraybackslash}m{1.715cm}|
		>{\centering\arraybackslash}m{1.715cm}|
		>{\centering\arraybackslash}m{1.715cm}|
		>{\centering\arraybackslash}m{1.715cm}|
		>{\centering\arraybackslash}m{1.715cm}|
		}
		\hline
		\rowcolor{gray!20}
		Sample Code & Length (m)           & Width (m)            & Height (m)             & Volume (m$^3$)            & Weight (Kg) & Density (Kg/m$^3$) \\ \hline
		CRB-01      & \multirow{3}{*}{0.2} & \multirow{3}{*}{0.1} & \multirow{3}{*}{0.027} & \multirow{3}{*}{0.00054 } & 0.50        & 925.92             \\
		\cline{1-1} \cline{6-7}
		CRB-02      &                      &                      &                        &                           & 0.56        & 1037.04            \\
		\cline{1-1} \cline{6-7}
		CRB-02      &                      &                      &                        &                           & 0.49        & 907.41             \\
		\cline{1-1} \cline{6-7}
		\hline
	\end{tabular}
	\caption{Density Test Result}
	\label{tab:placeholder}
\end{table}

\begin{figure}[H]
	\begin{minipage}{1\textwidth}
		\centering
		\fbox{\includegraphics[height=10cm, width=1\linewidth]{Assets/28.jpg}}
		\caption{Density Test Result}
	\end{minipage}
\end{figure}

\begin{enumerate}
	\item CRB-01 exhibited a density of 925.92 kg/m³, placing it in the mid-range among the three samples. The use of epoxy resin as the binder contributes to a relatively compact structure due to its high adhesive properties, enabling better encapsulation of textile waste particles. However, the density remains lower than that of CRB-02, indicating that while epoxy provides decent structural cohesion, it may leave some voids or have lower material packing efficiency compared to thermoplastics. The moderate density of CRB-01 suggests a balanced trade-off between weight and structural performance, making it suitable for non-load-bearing applications where lightweight and durability are desired.
	\item CRB-02 achieved the highest density value of 1037.04 kg/m³, clearly indicating the superior compactness of the material matrix. PET resin, being a thermoplastic, likely filled the voids more effectively and formed a denser matrix when combined with textile waste. The higher density implies a reduced porosity and greater material integrity, which may translate into better mechanical performance in terms of compressive and flexural strength. This makes CRB-02 potentially more suitable for structural applications where higher material density is advantageous, although thermal insulation properties may be compromised due to the reduced air pockets.
	\item CRB-03 recorded the lowest density of 907.41 kg/m³, suggesting a less compact and more porous internal structure compared to the other two variants. This outcome is likely due to the inherent lightweight nature of the plaster of Paris and the air entrainment typically associated with POP-cement mixtures. Although this lower density may result in reduced compressive strength, it could offer better thermal insulation and lower transportation costs. CRB-03 may therefore be more suited to interior partitioning, temporary structures, or applications where load-bearing capacity is not the primary concern but lightweight and insulation are desirable.
\end{enumerate}

% --------------- 4.1.3 --------------------%
\subsection{Surface Hardness Test Result}
\begin{table}[h!]
	\renewcommand{\arraystretch}{2} % row height
	\setlength{\tabcolsep}{20pt} % column padding
	\begin{tabular}{
		|>{\centering\arraybackslash}m{6cm}|
		>{\centering\arraybackslash}m{6cm}|
		}
		\hline
		\rowcolor{gray!20}
		Sample Code. & Surface Hardness (Shore D) \\ \hline
		CRB-01       & 78                         \\  \hline
		CRB-02       & 72                         \\  \hline
		CRB-02       & 58                         \\  \hline
	\end{tabular}
	\caption{Surface Hardness Test Result}
	\label{tab:placeholder}
\end{table}
\begin{enumerate}
	\item CRB-01 (Epoxy Brick) had the best surface hardness with a reading of 78 on the Shore D scale. This suggests a hard, stiff outer layer that is resistant to indentation and abrasion. The high hardness of epoxy is due to its robust cross-linked molecular structure, which makes the brick viable for use where surface durability matters. The high Shore D rating indicates the material's resistance to wear and mechanical damage, implying that CRB-01 may be suitable for use on exposed or high-contact surfaces like flooring tiles, wall panels, or cladding.
	\item CRB-02 (Polyester Brick) had a somewhat lower surface hardness of 72 Shore D, which is still in the high-performance range. Unsaturated polyester resin creates a tough matrix, although one that is a little less stiff than epoxy. The effect is a surface that retains good hardness and scratch resistance but permits a little more give under impact. This makes CRB-02 an excellent option for moderate-use construction surfaces, particularly where a compromise between toughness and workability is needed.
	\item 3. CRB-03 (Plaster of Paris Brick) had the lowest surface hardness, measuring Shore D 58. POP is softer and more brittle compared to resin-based composites and has a porous nature that predisposes it to higher surface wear and denting. Despite being reinforced with cement and shredded textiles, the surface is still less resistant to mechanical abrasion. Accordingly, CRB-03 is more appropriate for interior or decorative use where surface hardness is not a priority and physical contact is reduced.
\end{enumerate}
\begin{figure}[H]
	\begin{minipage}{1\textwidth}
		\centering
		\fbox{\includegraphics[height=10cm, width=1\linewidth]{Assets/29.png}}
		\caption{Surface Hardness Test Result}
	\end{minipage}
\end{figure}

% --------------- 4.1.4 --------------------%
\subsection{Flexural Strength Test Result}
\begin{table}[H]
	\renewcommand{\arraystretch}{2} % row height
	\setlength{\tabcolsep}{7pt} % column padding
	\begin{tabular}{
		|>{\centering\arraybackslash}m{2cm}|
		>{\centering\arraybackslash}m{2cm}|
		>{\centering\arraybackslash}m{2cm}|
		>{\centering\arraybackslash}m{2cm}|
		>{\centering\arraybackslash}m{2cm}|
		>{\centering\arraybackslash}m{2cm}|
		}
		\hline
		\rowcolor{gray!20}
		Sample Code. & Maximum Applied Load, P (kN) & Span Length, L (mm)  & Brick Width, b (mm)  & Depth, d (mm)       & Flexural Strength, R (MPa) \\ \hline
		CRB-01       & 1.25                         & \multirow{3}{*}{150} & \multirow{3}{*}{100} & \multirow{3}{*}{27} & 3.86                       \\
		\cline{1-2} \cline{6-6}
		CRB-02       & 1.15                         &                      &                      &                     & 3.55                       \\
		\cline{1-2} \cline{6-6}
		CRB-02       & 0.80                         &                      &                      &                     & 2.47                       \\
		\cline{1-2} \cline{6-6}
		\hline
	\end{tabular}
	\caption{Flexural Strength Test Result}
	\label{tab:placeholder}
\end{table}

\begin{figure}[H]
	\begin{minipage}{1\textwidth}
		\centering
		\fbox{\includegraphics[height=10cm, width=1\linewidth]{Assets/30.png}}
		\caption{Flexural Strength Test Result}
	\end{minipage}
\end{figure}

\begin{enumerate}
	\item CRB-01 (Epoxy Brick) was found to have a flexural strength of 3.86 MPa, demonstrating a high resistance to bending and cracking when under load. The good bonding of epoxy with shredded textiles enables stress to be efficiently transferred throughout the composite, enhancing the material's resistance to failure under the application of flexural forces. This qualifies CRB-01 for use in applications where bending stresses are prevalent, e.g., in thin wall panels, partition boards, or structural overlays, where surface continuity and durability are essential.
	\item CRB-02 (Polyester Brick) had a marginally lesser flexural strength of 3.55 MPa, which is still indicative of good performance. Polyester resin, although less rigid compared to epoxy, has superior ductility, which can be useful for resisting the propagation of cracks when subjected to bending. The presence of flexibility coupled with the reinforcement by shredded textiles enables the brick to absorb the bending loads without catastrophic fracture. The flexural capacity of CRB-02 recommends its application in lightweight structural elements or any surface area subjected to bending stress either during installation or in service.
	\item CRB-03 (Plaster of Paris Brick) showed the lowest flexural strength at 2.47 MPa, which is in line with the brittle nature of gypsum-based products. Despite the fact that the addition of shredded textiles and cement marginally increases the resistance of the brick to cracking, POP does not possess the structural toughness of resin-based systems inherently. Therefore, CRB-03 is more suitable for non-load-bearing uses with minimal flexural stress, e.g., decorative wall features or interior finishing items.
\end{enumerate}

% --------------- 4.1.5 --------------------%
\subsection{Water Absorption Test Result}
\begin{table}[H]
	\renewcommand{\arraystretch}{2} % row height
	\setlength{\tabcolsep}{10pt} % column padding
	\begin{tabular}{
		|>{\centering\arraybackslash}m{3cm}|
		>{\centering\arraybackslash}m{3cm}|
		>{\centering\arraybackslash}m{3cm}|
		>{\centering\arraybackslash}m{3cm}|
		}
		\hline
		\rowcolor{gray!20}
		Sample Code. & Weight before Submersion, W$_1$ (gm) & Weight after Submersion, W$_2$ (gm) & Water Absorption\% \\ \hline
		CRB-01       & 500                                  & 511.55                              & 2.31               \\ \hline
		CRB-02       & 560                                  & 581.78                              & 3.89               \\ \hline
		CRB-02       & 492                                  & 540.12                              & 9.78               \\ \hline
	\end{tabular}
	\caption{Water Absorption Test Result}
	\label{tab:placeholder}
\end{table}

\begin{figure}[H]
	\begin{minipage}{1\textwidth}
		\centering
		\fbox{\includegraphics[height=10cm, width=1\linewidth]{Assets/31.png}}
		\caption{Water Absorption Test Result}
	\end{minipage}
\end{figure}

\begin{enumerate}
	\item CRB-01 (Epoxy Brick) showed a water absorption rate of 2.31\%, which demonstrates the high resistance of epoxy resin to the penetration of water. Epoxy's tight, crosslinked network reduces the entry of water even in the presence of shredded textiles. This low absorption qualifies CRB-01 as a trusted material to be used in humid conditions or where water resistance is paramount.
	\item CRB-02 (Polyester Brick) had a marginally greater water absorption of 3.89\%. Although unsaturated polyester resin tends to be resistant to water, it is more permeable than epoxy, particularly when in composite form. The addition of shredded textiles and possible voids within the matrix are the cause of this modest degree of absorbency. CRB-02 remains appropriate for the majority of interior or semi-protected building uses but might need surface sealing in areas subjected to high moisture.
	\item CRB-03 (Plaster of Paris Brick) recorded the highest water absorption at 9.78\%, in line with the inherent porous and hygroscopic characteristics of gypsum-based products. Despite the incorporation of cement and shredded textiles, POP is prone to absorbing water easily, which can compromise its dimensional stability and durability in the long run. As it is, CRB-03 would be most suitable for use in dry interior areas or sheltered architectural elements where moisture exposure is kept to a bare minimum.
\end{enumerate}