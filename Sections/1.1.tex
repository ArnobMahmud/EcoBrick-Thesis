\section{Purpose and Significance of the Study}
The significance of this study lies in its multifaceted contribution: reducing the ecological footprint of textile waste, minimizing the environmental burden of traditional brick kilns, and introducing a cost-effective alternative building material tailored for the socio-economic and climatic context of Bangladesh. It also serves as a strategic alignment with national and global sustainable development goals, particularly those relating to waste management, climate action, affordable housing, and industry innovation. This study introduces a novel solution that addresses these twin challenges: the development of composite bricks reinforced with shredded textile waste (including cotton, polyester, denim, woven, and knitted fabrics) bound with resin. This alternative material not only diverts textile waste from landfills but also reduces dependency on clay extraction and fossil fuel consumption associated with traditional brick kilns. The use of thermosetting or bio-based resin as a binder enables effective encapsulation and solidification of shredded textiles, yielding bricks with favorable mechanical and thermal properties. The significance of this research lies in its multidimensional value: it proposes a technologically feasible, economically viable, and environmentally sustainable construction material. Furthermore, it aligns with Bangladesh's national development goals and global commitments, including the UN Sustainable Development Goals (SDGs) particularly those relating to responsible consumption, climate action, and sustainable cities. 