\section{Comparison with Traditional Fired Clay Brick (FCB)}

\renewcommand{\arraystretch}{2} % row height
\setlength{\tabcolsep}{7pt} % column padding
\begin{longtable}{
	|>{\centering\arraybackslash}m{2.5cm}|
	>{\centering\arraybackslash}m{2.5cm}|
	>{\centering\arraybackslash}m{2.5cm}|
	>{\centering\arraybackslash}m{2.5cm}|
	>{\centering\arraybackslash}m{2.5cm}|
	}
	\hline
	\rowcolor{gray!20}
	Property                   & CRB-01 (Epoxy)                & CRB-02 (Polyester)             & CRB-03 (POP)                    & Traditional Fired Clay Brick (FCB)      \\ \hline
	Compressive Strength (MPa) & 5.50                          & 6.50                           & 3.50                            & 12.50                                   \\ \hline
	Density (kg/m$^3$)         & 925.92                        & 1037.04                        & 907.41                          & 1750                                    \\ \hline
	Water Absorption (\%)      & 2.31                          & 3.89                           & 9.78                            & 17.50                                   \\ \hline
	Surface Hardness (Shore D) & 78                            & 72                             & 58                              & 85                                      \\ \hline
	Flexural Strength (MPa)    & 3.86                          & 3.55                           & 2.47                            & 4.10                                    \\ \hline
	Durability                 & High chemical resistance      & Good outdoor durability        & Limited (prone to water damage) & Prone to erosion, weathering            \\ \hline
	Eco friendliness           & Uses textile waste            & Uses textile waste             & Uses textile waste              & Energy intensive, high CO$_2$ emissions \\ \hline
	Construction Use           & Non-load earing walls, panels & Partition and light load walls & Decorative, false walls         & Load-bearing structural walls           \\ \hline
	\caption{Comparison with Traditional Clay Brick}
	\label{tab:placeholder}
\end{longtable}

\begin{figure}[H]
	\begin{minipage}{1\textwidth}
		\centering
		\fbox{\includegraphics[height=10cm, width=1\linewidth]{Assets/32.png}}
		\caption{Compressive Strength Comparison with Traditional Brick }
	\end{minipage}
\end{figure}

The compressive strength comparison overall indicates that conventional red brick possesses the highest value of 12.5 MPa, whereas the recycled textile-reinforced bricks CRB-02 (Polyester), CRB-01 (Epoxy), and CRB-03 (POP) yielded 6.5 MPa, 5.5 MPa, and 3.5 MPa, respectively. Even though these are inferior to conventional clay brick value, the recycled composite bricks possess some important advantages. They utilize shredded textile waste, aiding in solid waste minimization, reduced carbon footprint, and sustainable resources management. Moreover, these can be designed to be lighter in weight, which could help in the reduction of transportation cost and structural load, and the properties can be tailored based on the resin and fiber combination. Among the composites, the polyester-based CRB-02 was found to perform the best, and hence it has the potential to be used in non-load-bearing or partition wall applications. For improving their structural feasibility, improvement in material formulation - such as better fiber-matrix bonding and hybrid reinforcement is required. However, the environmental and economic advantages make textile-reinforced brick a promising alternative in green construction practices.

\begin{figure}[H]
	\begin{minipage}{1\textwidth}
		\centering
		\fbox{\includegraphics[height=10cm, width=1\linewidth]{Assets/33.png}}
		\caption{Density Comparison with Traditional Brick }
	\end{minipage}
\end{figure}

The bar graph displays the density values (kg/m$^3$) of four types of bricks: CRB-01 (Epoxy), CRB-02 (Polyester), CRB-03 (POP), and Traditional Clay Brick. Among the samples, traditional clay brick has the highest density at 1750.00 kg/m$^3$, being the heaviest and densest material. Conversely, the textile-reinforced composite bricks have very low densities, which can be useful in applications where light materials are required. CRB-02 (Polyester) is the most dense among the composites with a value of 1037.04 kg/m$^3$, followed by CRB-01 (Epoxy) with a value of 925.92 kg/m$^3$, and CRB-03 (POP) with 907.41 kg/m$^3$. The lightest material is CRB-03 and may provide benefits to structures that require lower dead loads. Whereas traditional bricks give higher mass and potentially more stability under load-bearing applications, the low density of composite bricks means easier handling, transportation, and possible thermal insulation advantages. Therefore, from the standpoint of weight efficiency, CRB-02 offers a good compromise between strength and density, while CRB-03 is better in terms of being the lightest, rendering these composite alternatives more appropriate for lightweight or modular building systems.

\begin{figure}[H]
	\begin{minipage}{1\textwidth}
		\centering
		\fbox{\includegraphics[height=10cm, width=1\linewidth]{Assets/34.jpg}}
		\caption{ Hardness Comparison with Traditional Brick}
	\end{minipage}
\end{figure}

The bar chart illustrating the Hardness (Shore D) of different materials provides valuable insight into their surface durability and resistance to deformation, which are critical for structural and wear-intensive applications. Traditional clay bricks again show the highest hardness value at 85 Shore D, reinforcing their established reputation for robustness and long-term durability in construction. Among the composite alternatives, Epoxy-based bricks (CRB-01) exhibit a high hardness of 78 Shore D, suggesting excellent resistance to surface wear, scratches, and minor impacts, making them a strong candidate for replacing traditional materials in both load-bearing and surface-exposed applications. Polyester-based bricks (CRB-02) follow with a moderate hardness of 72 Shore D, indicating acceptable durability but with slightly reduced performance under abrasive or impact conditions. POP-based bricks (CRB-03) record the lowest hardness at 58 Shore D, highlighting their soft and brittle nature, which limits their practical use to decorative or low-impact applications. Overall, the data strongly supports the use of epoxy as the most effective binder among the tested alternatives when hardness and surface resilience are key performance criteria in composite brick manufacturing.

\begin{figure}[H]
	\begin{minipage}{1\textwidth}
		\centering
		\fbox{\includegraphics[height=10cm, width=1\linewidth]{Assets/35.jpg}}
		\caption{Flexural Strength Comparison with Traditional Brick }
	\end{minipage}
\end{figure}

The bar chart comparing the Flexural Strength (MPa) of various materials reveals critical insights into the mechanical performance of each binder used in composite brick production. Traditional clay bricks exhibit the highest flexural strength at 4.10 MPa, indicating their superior load-bearing capability, which aligns with their long-standing use in construction. Among the composite alternatives, Epoxy-based bricks (CRB-01) demonstrate the highest strength at 3.86 MPa, closely rivaling traditional clay, and suggesting that epoxy offers excellent bonding and structural integrity within the composite matrix. Polyester-based bricks (CRB-02) follow with a slightly lower strength of 3.55 MPa, still within a usable range but indicating slightly less rigidity and bonding efficiency compared to epoxy. In contrast, Plaster of Paris (POP)-based bricks (CRB-03) show significantly lower flexural strength at 2.47 MPa, which suggests limited structural performance and suitability only for non-load-bearing or decorative applications. These results suggest that epoxy resin is the most viable alternative binder for achieving high structural performance in eco-friendly composite bricks, closely emulating or even surpassing traditional clay when optimized further.

\begin{figure}[H]
	\begin{minipage}{1\textwidth}
		\centering
		\fbox{\includegraphics[height=10cm, width=1\linewidth]{Assets/36.png}}
		\caption{Water Absorption Comparison with Traditional Brick }
	\end{minipage}
\end{figure}

The results of the water absorption test indicate considerable variation amongst the four types of bricks. CRB-01 (Epoxy) gave the lowest water absorption of 2.31\%, which is the best amongst all the samples, making the material the most resistant to water. This is due to the fact that the dense and impermeable nature of epoxy's polymer network effectively keeps moisture out. Next, CRB-02 (Polyester) presented a slightly higher but still low absorption of 3.89\%, proving resistant to water and the second-best performing material. CRB-03 (POP) presented a considerably higher absorption of 9.78\%, with only moderate resistance to water but still an improvement compared to conventional clay brick. Traditional clay brick, on the other hand, gave the highest absorption of 17.50\%, demonstrating its porous nature and susceptibility to moistureinduced damage. Of the four, CRB-01 is by far the most resistant to water, followed by CRB-02, while CRB-03, despite being less efficient, still outperforms conventional clay brick. The findings indicate that the recycled textile-reinforced composite bricks, specifically the epoxy or polyester resin-based ones, provide enhanced protection from water penetration, making them more durable and suitable for application in damp or moisture-susceptible areas.