\section{General Fabrication Process of Composite Bricks}
The fabrication of textile-reinforced composite bricks involved a multi-step process including mold preparation, resin mixing, reinforcement integration, casting, and curing. The following steps were followed for consistency and accuracy across samples:

% --------------- 3.3.1 --------------------%
\subsection{Mold Preparation}
\begin{enumerate}
	\item  Material
	      \begin{itemize}
		      \item Wooden molds of dimension 200 mm $\times$ 100 mm $\times$ 60 mm were used.
	      \end{itemize}
	\item Cleaning
	      \begin{itemize}
		      \item Molds were thoroughly cleaned to remove any debris or residual materials from previous castings.
	      \end{itemize}
	\item  Release Agent
	      \begin{itemize}
		      \item A thin coat of mold release agent (e.g., petroleum jelly or silicone spray) was applied to the inner surfaces of the mold. This ensures easy de-molding of the cured brick and prevents bonding with the metal surface.
	      \end{itemize}
	\item  Alignment
	      \begin{itemize}
		      \item The mold parts were tightly assembled and checked for alignment to avoid leakage or deformations during casting.
	      \end{itemize}
\end{enumerate}

\begin{figure}[H]
	\centering
	\includegraphics[width=\linewidth]{Assets/10.jpg}
	\caption{Mold}
	\label{fig:placeholder}
\end{figure}

% --------------- 3.3.2 --------------------%
\subsection{Chemical Mixing}
\begin{enumerate}
	\item  Type
	      \begin{itemize}
		      \item Epoxy, Polyester resin and POP was selected for its strong adhesive and mechanical properties.
	      \end{itemize}
	\item Ratio
	      \begin{itemize}
		      \item he resin and hardener were mixed in the manufacturer-specified ratio, typically 2:1 (resin:hardener) by weight.
	      \end{itemize}
	\item  Mixing
	      \begin{itemize}
		      \item The components were thoroughly stirred in a clean, dry container using a plastic or wooden stirrer for 3 - 5 minutes until a uniform, bubble-free mixture was obtained.
	      \end{itemize}
	\item  Precaution
	      \begin{itemize}
		      \item Mixing was done slowly to minimize air entrapment and premature curing.
	      \end{itemize}
\end{enumerate}

\begin{figure}[H]
	\centering
	\includegraphics[height=14cm, width=.7\linewidth]{Assets/11.jpg}
	\caption{Chemical Mixing}
	\label{fig:placeholder}
\end{figure}

% --------------- 3.3.3 --------------------%
\subsection{Addition of Chopped Textile Waste}
\begin{enumerate}
	\item  Quantity
	      \begin{itemize}
		      \item Exactly 100 grams of chopped textile waste (cotton, polyester, denim, woven, and knitted fabric) was used per mold.
	      \end{itemize}
	\item Preparation
	      \begin{itemize}
		      \item Waste fabric was shredded into 15 - 30 mm long shredded textiles using a mechanical shredder.
		      \item The shredded textiles were dried in sunlight or a hot-air oven at 60 - 80°C to remove residual moisture.
	      \end{itemize}
	\item  Integration
	      \begin{itemize}
		      \item The shredded textiles were gradually added to the mixed resin.
		      \item Continuous stirring was done for 5 - 7 minutes to ensure even dispersion and full wetting of shredded textiles.
		      \item Care was taken to avoid clumping or fiber balling.
	      \end{itemize}
\end{enumerate}

\begin{figure}[H]
	\centering
	\begin{minipage}{0.48\textwidth}
		\centering
		\includegraphics[width=1\linewidth]{Assets/12.jpg}
		\caption{Shredded Textile Waste Measuring}
	\end{minipage}
	\hfill
	\begin{minipage}{0.48\textwidth}
		\centering
		\includegraphics[width=1\linewidth]{Assets/13.jpg}
		\caption{Shredded Textile Mix with Chemical}
	\end{minipage}
\end{figure}

\newpage{}

% --------------- 3.3.4 --------------------%
\subsection{Casting into Mold}
\begin{enumerate}
	\item  Pouring
	      \begin{itemize}
		      \item The composite paste was slowly poured into the mold cavity.
	      \end{itemize}
	\item  Leveling and Compaction
	      \begin{itemize}
		      \item A spatula or trowel was used to spread the mixture evenly across the mold.
		      \item Gentle manual tapping or mechanical vibration was applied to eliminate trapped air bubbles and ensure compactness.
	      \end{itemize}
	\item Surface Finish
	      \begin{itemize}
		      \item The top surface was smoothed using a trowel or lid.
		      \item Excess mix was scraped off to maintain a consistent top edge.
	      \end{itemize}
	\item Pressure
	      \begin{itemize}
		      \item Mold was kept under pressure of about 14 psi for 2 hours at RT.
		      \item Load was removed after completion of curing time.
	      \end{itemize}
\end{enumerate}

\begin{figure}[H]
	\centering
	\begin{minipage}{0.48\textwidth}
		\centering
		\includegraphics[height=9cm, width=1\linewidth]{Assets/14.jpg}
	\end{minipage}
	\hfill
	\begin{minipage}{0.48\textwidth}
		\centering
		\includegraphics[height=9cm, width=1\linewidth]{Assets/15.jpg}
	\end{minipage}
	\caption{Casting into Mold}
\end{figure}

% --------------- 3.3.5 --------------------%
\subsection{Curing Process}
\begin{enumerate}
	\item  Initial Curing
	      \begin{itemize}
		      \item  The mold was left undisturbed at room temperature (25° - 30°C) for a minimum of 2 - 4 hours to allow full chemical curing of the epoxy.
	      \end{itemize}
	\item  De-molding
	      \begin{itemize}
		      \item After curing, bricks were carefully removed from the molds.
		      \item If necessary, light tapping with a rubber mallet was used to release the brick without cracking.
	      \end{itemize}
	\item Post-Curing (optional)
	      \begin{itemize}
		      \item Bricks were air-dried for an additional 1 – 2 days to stabilize mechanical properties and eliminate any trapped volatiles.
	      \end{itemize}
\end{enumerate}

\begin{figure}[H]
	\centering
	\begin{minipage}{0.48\textwidth}
		\centering
		\includegraphics[width=1\linewidth]{Assets/16.jpg}
	\end{minipage}
	\hfill
	\begin{minipage}{0.48\textwidth}
		\centering
		\includegraphics[width=1\linewidth]{Assets/17.jpg}
	\end{minipage}
	\caption{Curing}
\end{figure}

% --------------- 3.3.6 --------------------%
\subsection{Labeling and Storage}
\begin{enumerate}
	\item Each sample was assigned a unique code (e.g., CRB – Composite Reinforce Brick).
	\item Bricks were stored in a dry, dust-free environment before testing.
\end{enumerate}

\begin{figure}[H]
	\centering
	\begin{minipage}{0.48\textwidth}
		\centering
		\includegraphics[width=1\linewidth]{Assets/18.jpg}
	\end{minipage}
	\hfill
	\begin{minipage}{0.48\textwidth}
		\centering
		\includegraphics[width=1\linewidth]{Assets/19.jpg}
	\end{minipage}
	\caption{Labeling}
\end{figure}

% --------------- 3.3.7 --------------------%
\subsection{Safety Measures}
\begin{enumerate}
	\item PPE Used: Gloves, masks, and eye protection.
	\item Ventilation: Work was done in an open or exhaust-ventilated area to avoid inhalation of MEKP fumes.
	\item Waste Disposal:
	      \begin{itemize}
		      \item Unused resin was disposed of in sealed containers.
		      \item Contaminated rags and gloves were placed in dedicated waste bins for hazardous materials.
	      \end{itemize}
\end{enumerate}
\newpage{}
\underline{Summary:}

\begin{table}[h!]
	\renewcommand{\arraystretch}{2} % row height
	\setlength{\tabcolsep}{10pt} % column padding
	\begin{tabular}{|>{\centering\arraybackslash}m{2cm}|>{\centering\arraybackslash}m{5cm}|>{\centering\arraybackslash}m{5cm}|}
		\hline
		\rowcolor{gray!20}
		SL No. & Parameter                & Details                                       \\
		\hline
		1      & Textile Waste Used       & Cotton, Polyester, Denim, Woven, Knitted      \\
		\hline
		2      & Textile Weight per Brick & 100 grams                                     \\
		\hline
		3      & Resin Type               & Epoxy resin (2:1 ratio with hardener)         \\
		\hline
		4      & Mold Size                & 200 mm $\times$ 100 mm $\times$  60 mm        \\
		\hline
		5      & Mixing Time              & 5 - 7 minutes (manual or mechanical)          \\
		\hline
		6      & Curing Time              & 2 - 4 hours (initial), 1 - 2  days (optional) \\
		\hline
		7      & Environment              & Ambient (25 - 30°C), well-ventilated space    \\
		\hline
	\end{tabular}
	\caption{Fabrication Process Summary}
	\label{tab:placeholder}
\end{table}