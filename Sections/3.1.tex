\section{Materials Used }

% --------------- 3.1.1 --------------------%
\subsection{Textile Waste}
The textile waste used in this study was collected from garment factories and tailoring units located in Jamalpur, Gazipur and Dhaka. The selected categories included: 

\begin{enumerate}
    \item Cotton (natural fiber scraps)
    \item Polyester (synthetic waste)
    \item Denim (mixed fiber, predominantly cotton/polyester blend)
    \item Woven fabrics (interlaced yarn waste)
    \item Knitted fabrics (looped yarn waste)
\end{enumerate}

\noindent All waste fabrics were manually sorted, cleaned, and shredded into uniform shredded textiles (approximately 15 - 30 mm in length) using a mechanical shredder to ensure compatibility with the binder matrix. To assess the consistency of waste size, a random sample of 10 shredded textile waste was selected from the prepared reinforcement material. The goal was to determine the average wastage length, understand the degree of variation, and confirm the suitability of the chopped textiles size for composite brick fabrication.
\vspace{12pt}
\begin{figure}[H]
    \centering
    \includegraphics[width=.75\linewidth]{Assets/4.png}
    \caption{Textile Waste Scraps}
    \label{fig:placeholder}
\end{figure}

\begin{figure}[H]
    \centering
    \includegraphics[width=.6\linewidth]{Assets/5.jpg}
    \caption{Textile Shredding Machine}
    \label{fig:placeholder}
\end{figure}

\begin{figure}[H]
    \centering
    \includegraphics[width=.6\linewidth]{Assets/6.jpg}
    \caption{Shredded Textile Waste}
    \label{fig:placeholder}
\end{figure}

\newpage{}
\textbf{Sampling Method }
\begin{enumerate}
    \item A random sampling method was used to minimize bias.
    \item 10 shredded textiles were picked from the mixed batch (cotton, polyester, denim, woven, knitted).
    \item Each length was measured using a millimeter scale or digital caliper.
\end{enumerate}


\begin{table}[h!]
\renewcommand{\arraystretch}{2} % row height
\setlength{\tabcolsep}{10pt} % column padding
\begin{tabular}{|>{\centering\arraybackslash}m{2cm}|>{\centering\arraybackslash}m{2cm}|>{\centering\arraybackslash}m{2cm}|>{\centering\arraybackslash}m{2cm}|> {\centering\arraybackslash}m{2cm}|}
\hline
    \rowcolor{gray!20}
         Sample No. & Length(mm) & Mean(mm) & SD & CV\% \\
         \hline
        SRD-01 & 15 & \multirow{10}{*}{18.9} & \multirow{10}{*}{3.75} & \multirow{10}{*}{19.84\%} \\
        \cline{1-2}
        SRD-02 & 18 &  &  &  \\ 
        \cline{1-2}
        SRD-03 & 21 &  &  &  \\
        \cline{1-2}
        SRD-04 & 25 &  &  &  \\
        \cline{1-2}
        SRD-05 & 15 &  &  &  \\
        \cline{1-2}
        SRD-06 & 17 &  &  &  \\
        \cline{1-2}
        SRD-07 & 25 &  &  &  \\
        \cline{1-2}
        SRD-08 & 17 &  &  &  \\
        \cline{1-2}
        SRD-09 & 16 &  &  &  \\
        \cline{1-2}
        SRD-10 & 20 &  &  &  \\
        \hline
    \end{tabular}
    \caption{Sampling}
    \label{tab:placeholder}
\end{table}

\noindent These values indicate a moderate variation in lengths, which is generally acceptable for reinforcement in composite materials, as it supports a good balance between reinforcement dispersion and mechanical bonding. 

\begin{figure}[H]
    \centering
    \begin{minipage}{0.45\textwidth}
    \centering
    \fbox{\includegraphics[height=5cm, width=1\linewidth]{Assets/7.png}}
    \caption{Boxplot of Textile Sample Lengths}
    \end{minipage}
    \hfill
    \begin{minipage}{0.45\textwidth}
    \centering
    \fbox{\includegraphics[height=5cm, width=1\linewidth]{Assets/8.png}}
    \caption{SD and CV of Textile Samples Length}
    \end{minipage}
\end{figure}

\begin{figure}[H]
    \centering
    \fbox{\includegraphics[width=1\linewidth]{Assets/9.png}}
    \caption{Histogram of Textile Samples Length }
\end{figure}

% --------------- 3.1.2 --------------------%
\subsection{Binders}
Three binding agents were used to evaluate bonding performance and cost-efficiency: 
\begin{enumerate}
    \item Polyester Resin
    \begin{itemize}
        \item Industrial-grade, two-part system (resin + hardener) Used due to high bonding strength and durability
    \end{itemize}
    \item Epoxy Resin
    \begin{itemize}
        \item Used for its quick setting and cost-efficiency
    \end{itemize}
    \item Plaster of Paris (POP)
    \begin{itemize}
        \item Used in alternative mixes for comparison 
        \item Provided a brittle, quick-curing matrix 
    \end{itemize}
\end{enumerate}

% --------------- 3.1.3 --------------------%
\subsection{Additional Materials}
\begin{enumerate}
    \item Hardener
    \begin{itemize}
        \item Mixed with polyester and epoxy resin to initiate curing. 
    \end{itemize}
    \item Water
    \begin{itemize}
        \item Added in measured quantity during POP-based mixes. 
    \end{itemize}
    \item  Silica Sand/ Cement (optional)
    \begin{itemize}
        \item Used in some mixes as a filler for texture and bulk. 
    \end{itemize}
\end{enumerate}

% --------------- 3.1.4 --------------------%
\subsection{Mold}
\begin{enumerate}
    \item Material
    \begin{itemize}
        \item Wood. 
    \end{itemize}
    \item Dimension
    \begin{itemize}
        \item 200 mm × 100 mm × 60 mm. 
    \end{itemize}
    \item  Features
    \begin{itemize}
        \item Custom-built for this research to allow extended casting and segment cutting 
        \item Inner surfaces were coated with mold release agent (oil or wax) to prevent sticking 
        \item Sturdy and reusable, fastened with screws and braced for shape retention 
    \end{itemize}
\end{enumerate}

% --------------- 3.1.5 --------------------%
\subsection{Additives and Tools}
\begin{enumerate}
    \item  Mold Release Agent 
    \begin{itemize}
        \item Petroleum jelly or silicone-based lubricant. 
    \end{itemize}
    \item Stirring Tools 
    \begin{itemize}
        \item Wooden sticks, mechanical mixer.  
    \end{itemize}
    \item  Cutting Tools 
    \begin{itemize}
        \item Scissors, shredders, and utility blades.
    \end{itemize}
    \item  Protective Gear 
    \begin{itemize}
        \item Gloves, and face masks for safe resin handling.
    \end{itemize}
\end{enumerate}