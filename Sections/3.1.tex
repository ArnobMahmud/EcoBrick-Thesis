\section{Materials Used}
\subsection{Textile Waste}
The textile waste used in this study was collected from garment factories and tailoring units located in Jamalpur, Gazipur and Dhaka. The selected categories included:

\begin{figure}[ht]
	\centering
	\begin{minipage}{0.30\textwidth}
		\centering
		\includegraphics[height=6cm, width=1\linewidth]{Assets/1.png}
		\caption{Epoxy Resin Brick Sample}
	\end{minipage}
	\hfill
	\begin{minipage}{0.30\textwidth}
		\centering
		\includegraphics[height=6cm, width=1\linewidth]{Assets/2.png}
		\caption{PET Resin Brick Sample}
	\end{minipage}
	\hfill
	\begin{minipage}{0.30\textwidth}
		\centering
		\includegraphics[height=6cm, width=1\linewidth]{Assets/3.png}
		\caption{POP Resin Brick Sample}
	\end{minipage}
\end{figure}

\begin{enumerate}
	\item Cotton (natural fiber scraps)
	\item Polyester (synthetic waste)
	\item Denim (mixed fiber, predominantly cotton/polyester blend)
	\item Woven fabrics (interlaced yarn waste)
	\item Knitted fabrics (looped yarn waste)
\end{enumerate}
All waste fabrics were manually sorted, cleaned, and shredded into uniform shredded textiles (approximately 15 - 30 mm in length) using a mechanical shredder to ensure compatibility with the binder matrix. To assess the consistency of waste size, a random sample of 10 shredded textile waste was selected from the prepared reinforcement material. The goal was to determine the average wastage length, understand the degree of variation, and confirm the suitability of the chopped textiles size for composite brick fabrication.